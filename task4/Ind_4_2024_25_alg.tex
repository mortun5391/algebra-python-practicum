 \documentclass[11pt]{report}

\usepackage[T2A]{fontenc}

\usepackage[utf8]{inputenc}

\usepackage[russian]{babel}

\usepackage{amsmath,amssymb}

\usepackage{graphicx}

\oddsidemargin=-19mm

\topmargin=-30mm

\textheight 26cm 

\hsize 18cm

\textwidth 20cm

\begin{document}

\pagestyle{empty}

{\bf Индивидуальное задание 4.}

Вариант N 1
Дана СЛАУ $AX = b$,
Проверить совместность по теореме Кронекера-Капелли. Если СЛАУ совместна, проверить единственность решения.
Для соответствующей однородной СЛАУ проверить существование нетривиального решения. В случае, если оно существует,
найти размерность пространства решений и составить ФСР и общее решение однородной  и неоднородной СЛАУ.


\begin{align*}
 A = \left[\begin{matrix}-9 & -9 & -5 & -8 & 2\\0 & -3 & 0 & -5 & 6\\-36 & -48 & -20 & -52 & 32\\-36 & -24 & -20 & -12 & -16\\-7 & -8 & 3 & 2 & -1\end{matrix}\right],
\ b = \left[\begin{matrix}-34\\-46\\-320\\48\\74\end{matrix}\right]. 
 \end{align*}

Вариант N 2
Дана СЛАУ $AX = b$,
Проверить совместность по теореме Кронекера-Капелли. Если СЛАУ совместна, проверить единственность решения.
Для соответствующей однородной СЛАУ проверить существование нетривиального решения. В случае, если оно существует,
найти размерность пространства решений и составить ФСР и общее решение однородной  и неоднородной СЛАУ.


\begin{align*}
 A = \left[\begin{matrix}-6 & -4 & -2 & -3 & 0\\-2 & -8 & -1 & 3 & 8\\-14 & 24 & -3 & -27 & -40\\8 & 8 & -9 & 8 & 0\\-4 & 4 & -2 & 4 & -2\end{matrix}\right],
\ b = \left[\begin{matrix}2\\38\\-182\\80\\-22\end{matrix}\right]. 
 \end{align*}

Вариант N 3
Дана СЛАУ $AX = b$,
Проверить совместность по теореме Кронекера-Капелли. Если СЛАУ совместна, проверить единственность решения.
Для соответствующей однородной СЛАУ проверить существование нетривиального решения. В случае, если оно существует,
найти размерность пространства решений и составить ФСР и общее решение однородной  и неоднородной СЛАУ.


\begin{align*}
 A = \left[\begin{matrix}-8 & -2 & -6 & 2 & 2\\-68 & 24 & -56 & 0 & -28\\-304 & 88 & -248 & 8 & -104\\240 & -104 & 200 & 8 & 120\\752 & -304 & 624 & 16 & 352\end{matrix}\right],
\ b = \left[\begin{matrix}-14\\-512\\-2104\\1992\\6032\end{matrix}\right]. 
 \end{align*}

Вариант N 4
Дана СЛАУ $AX = b$,
Проверить совместность по теореме Кронекера-Капелли. Если СЛАУ совместна, проверить единственность решения.
Для соответствующей однородной СЛАУ проверить существование нетривиального решения. В случае, если оно существует,
найти размерность пространства решений и составить ФСР и общее решение однородной  и неоднородной СЛАУ.


\begin{align*}
 A = \left[\begin{matrix}0 & -7 & 1 & -8 & 2\\2 & -3 & 5 & 2 & 0\\10 & -43 & 29 & -22 & 8\\-10 & -13 & -21 & -42 & 8\\-5 & -2 & 6 & 1 & -4\end{matrix}\right],
\ b = \left[\begin{matrix}87\\-7\\313\\383\\48\end{matrix}\right]. 
 \end{align*}

Вариант N 5
Дана СЛАУ $AX = b$,
Проверить совместность по теореме Кронекера-Капелли. Если СЛАУ совместна, проверить единственность решения.
Для соответствующей однородной СЛАУ проверить существование нетривиального решения. В случае, если оно существует,
найти размерность пространства решений и составить ФСР и общее решение однородной  и неоднородной СЛАУ.


\begin{align*}
 A = \left[\begin{matrix}-7 & -8 & 0 & -1 & 9\\-7 & 5 & -7 & 7 & 7\\0 & -52 & 28 & -32 & 8\\-4 & -1 & -3 & 6 & 7\\-7 & -2 & 4 & -4 & 5\end{matrix}\right],
\ b = \left[\begin{matrix}-27\\46\\-292\\41\\-98\end{matrix}\right]. 
 \end{align*}

Вариант N 6
Дана СЛАУ $AX = b$,
Проверить совместность по теореме Кронекера-Капелли. Если СЛАУ совместна, проверить единственность решения.
Для соответствующей однородной СЛАУ проверить существование нетривиального решения. В случае, если оно существует,
найти размерность пространства решений и составить ФСР и общее решение однородной  и неоднородной СЛАУ.


\begin{align*}
 A = \left[\begin{matrix}-1 & -1 & -4 & 3 & -9\\-7 & 3 & 5 & -1 & 7\\-32 & 8 & 4 & 8 & -8\\24 & -16 & -36 & 16 & -64\\8 & 8 & -9 & 4 & 6\end{matrix}\right],
\ b = \left[\begin{matrix}53\\-86\\-132\\556\\-71\end{matrix}\right]. 
 \end{align*}

Вариант N 7
Дана СЛАУ $AX = b$,
Проверить совместность по теореме Кронекера-Капелли. Если СЛАУ совместна, проверить единственность решения.
Для соответствующей однородной СЛАУ проверить существование нетривиального решения. В случае, если оно существует,
найти размерность пространства решений и составить ФСР и общее решение однородной  и неоднородной СЛАУ.


\begin{align*}
 A = \left[\begin{matrix}3 & -6 & 7 & -8 & 6\\-13 & 21 & 8 & -12 & 49\\-53 & 81 & 68 & -92 & 269\\77 & -129 & -12 & 28 & -221\\219 & -363 & -64 & 116 & -687\end{matrix}\right],
\ b = \left[\begin{matrix}-78\\-277\\-1697\\1073\\3531\end{matrix}\right]. 
 \end{align*}

Вариант N 8
Дана СЛАУ $AX = b$,
Проверить совместность по теореме Кронекера-Капелли. Если СЛАУ совместна, проверить единственность решения.
Для соответствующей однородной СЛАУ проверить существование нетривиального решения. В случае, если оно существует,
найти размерность пространства решений и составить ФСР и общее решение однородной  и неоднородной СЛАУ.


\begin{align*}
 A = \left[\begin{matrix}9 & -1 & -9 & 0 & 4\\3 & -4 & 3 & 1 & 0\\15 & 13 & -39 & -4 & 12\\5 & 6 & -5 & -9 & -2\\-2 & -9 & -2 & 9 & -8\end{matrix}\right],
\ b = \left[\begin{matrix}-87\\-38\\-109\\10\\-137\end{matrix}\right]. 
 \end{align*}

Вариант N 9
Дана СЛАУ $AX = b$,
Проверить совместность по теореме Кронекера-Капелли. Если СЛАУ совместна, проверить единственность решения.
Для соответствующей однородной СЛАУ проверить существование нетривиального решения. В случае, если оно существует,
найти размерность пространства решений и составить ФСР и общее решение однородной  и неоднородной СЛАУ.


\begin{align*}
 A = \left[\begin{matrix}1 & 5 & 4 & -7 & 6\\3 & -5 & -5 & 5 & 8\\15 & -5 & -8 & -1 & 50\\-9 & 35 & 32 & -41 & -14\\-2 & 7 & 3 & -9 & 2\end{matrix}\right],
\ b = \left[\begin{matrix}107\\-103\\-91\\733\\121\end{matrix}\right]. 
 \end{align*}

Вариант N 10
Дана СЛАУ $AX = b$,
Проверить совместность по теореме Кронекера-Капелли. Если СЛАУ совместна, проверить единственность решения.
Для соответствующей однородной СЛАУ проверить существование нетривиального решения. В случае, если оно существует,
найти размерность пространства решений и составить ФСР и общее решение однородной  и неоднородной СЛАУ.


\begin{align*}
 A = \left[\begin{matrix}-8 & 9 & -7 & 9 & 8\\-4 & -7 & 0 & -5 & -2\\-44 & -8 & -21 & 2 & 14\\-4 & 62 & -21 & 52 & 34\\8 & -3 & 7 & -4 & -8\end{matrix}\right],
\ b = \left[\begin{matrix}5\\32\\175\\-145\\-9\end{matrix}\right]. 
 \end{align*}

Вариант N 11
Дана СЛАУ $AX = b$,
Проверить совместность по теореме Кронекера-Капелли. Если СЛАУ совместна, проверить единственность решения.
Для соответствующей однородной СЛАУ проверить существование нетривиального решения. В случае, если оно существует,
найти размерность пространства решений и составить ФСР и общее решение однородной  и неоднородной СЛАУ.


\begin{align*}
 A = \left[\begin{matrix}5 & -4 & -4 & 5 & 3\\55 & -27 & -2 & 20 & 24\\290 & -147 & -22 & 115 & 129\\-260 & 123 & -2 & -85 & -111\\-795 & 381 & 6 & -270 & -342\end{matrix}\right],
\ b = \left[\begin{matrix}43\\389\\2074\\-1816\\-5577\end{matrix}\right]. 
 \end{align*}

Вариант N 12
Дана СЛАУ $AX = b$,
Проверить совместность по теореме Кронекера-Капелли. Если СЛАУ совместна, проверить единственность решения.
Для соответствующей однородной СЛАУ проверить существование нетривиального решения. В случае, если оно существует,
найти размерность пространства решений и составить ФСР и общее решение однородной  и неоднородной СЛАУ.


\begin{align*}
 A = \left[\begin{matrix}6 & 0 & 6 & -3 & 1\\9 & 5 & 9 & -7 & -1\\-18 & -20 & -18 & 19 & 7\\3 & -5 & 1 & -4 & 5\\8 & 4 & -1 & 3 & -5\end{matrix}\right],
\ b = \left[\begin{matrix}-2\\7\\-34\\-60\\55\end{matrix}\right]. 
 \end{align*}

Вариант N 13
Дана СЛАУ $AX = b$,
Проверить совместность по теореме Кронекера-Капелли. Если СЛАУ совместна, проверить единственность решения.
Для соответствующей однородной СЛАУ проверить существование нетривиального решения. В случае, если оно существует,
найти размерность пространства решений и составить ФСР и общее решение однородной  и неоднородной СЛАУ.


\begin{align*}
 A = \left[\begin{matrix}5 & 4 & 5 & 2 & -2\\20 & 28 & 0 & 40 & -44\\100 & 128 & 20 & 168 & -184\\-60 & -96 & 20 & -152 & 168\\-200 & -304 & 40 & -464 & 512\end{matrix}\right],
\ b = \left[\begin{matrix}-34\\-344\\-1512\\1240\\3856\end{matrix}\right]. 
 \end{align*}

Вариант N 14
Дана СЛАУ $AX = b$,
Проверить совместность по теореме Кронекера-Капелли. Если СЛАУ совместна, проверить единственность решения.
Для соответствующей однородной СЛАУ проверить существование нетривиального решения. В случае, если оно существует,
найти размерность пространства решений и составить ФСР и общее решение однородной  и неоднородной СЛАУ.


\begin{align*}
 A = \left[\begin{matrix}-1 & -6 & 6 & -8 & 4\\5 & -9 & -9 & 6 & -2\\17 & -54 & -18 & 0 & 4\\-23 & 18 & 54 & -48 & 20\\1 & -1 & -9 & -5 & -4\end{matrix}\right],
\ b = \left[\begin{matrix}135\\-38\\253\\557\\-56\end{matrix}\right]. 
 \end{align*}

Вариант N 15
Дана СЛАУ $AX = b$,
Проверить совместность по теореме Кронекера-Капелли. Если СЛАУ совместна, проверить единственность решения.
Для соответствующей однородной СЛАУ проверить существование нетривиального решения. В случае, если оно существует,
найти размерность пространства решений и составить ФСР и общее решение однородной  и неоднородной СЛАУ.


\begin{align*}
 A = \left[\begin{matrix}6 & 2 & 8 & 6 & 6\\-6 & 3 & 32 & -21 & -1\\-6 & 23 & 192 & -81 & 19\\54 & -7 & -128 & 129 & 29\\138 & -29 & -416 & 363 & 63\end{matrix}\right],
\ b = \left[\begin{matrix}-28\\-192\\-1072\\848\\2656\end{matrix}\right]. 
 \end{align*}

Вариант N 16
Дана СЛАУ $AX = b$,
Проверить совместность по теореме Кронекера-Капелли. Если СЛАУ совместна, проверить единственность решения.
Для соответствующей однородной СЛАУ проверить существование нетривиального решения. В случае, если оно существует,
найти размерность пространства решений и составить ФСР и общее решение однородной  и неоднородной СЛАУ.


\begin{align*}
 A = \left[\begin{matrix}9 & 1 & -8 & 7 & -6\\3 & 0 & 9 & -7 & -7\\39 & 3 & 12 & -7 & -46\\15 & 3 & -60 & 49 & 10\\3 & -7 & 6 & 4 & 6\end{matrix}\right],
\ b = \left[\begin{matrix}-29\\3\\-75\\-99\\-32\end{matrix}\right]. 
 \end{align*}

Вариант N 17
Дана СЛАУ $AX = b$,
Проверить совместность по теореме Кронекера-Капелли. Если СЛАУ совместна, проверить единственность решения.
Для соответствующей однородной СЛАУ проверить существование нетривиального решения. В случае, если оно существует,
найти размерность пространства решений и составить ФСР и общее решение однородной  и неоднородной СЛАУ.


\begin{align*}
 A = \left[\begin{matrix}9 & -4 & -2 & 0 & -4\\-2 & 5 & 9 & -1 & 4\\44 & -36 & -44 & 4 & -32\\-2 & 1 & -9 & 5 & -4\\0 & 8 & 5 & -9 & 9\end{matrix}\right],
\ b = \left[\begin{matrix}-33\\41\\-296\\-21\\47\end{matrix}\right]. 
 \end{align*}

Вариант N 18
Дана СЛАУ $AX = b$,
Проверить совместность по теореме Кронекера-Капелли. Если СЛАУ совместна, проверить единственность решения.
Для соответствующей однородной СЛАУ проверить существование нетривиального решения. В случае, если оно существует,
найти размерность пространства решений и составить ФСР и общее решение однородной  и неоднородной СЛАУ.


\begin{align*}
 A = \left[\begin{matrix}7 & 0 & -2 & -2 & -1\\0 & 2 & 1 & 9 & 9\\21 & -10 & -11 & -51 & -48\\-8 & 4 & 2 & 2 & -3\\7 & -1 & 1 & -3 & 4\end{matrix}\right],
\ b = \left[\begin{matrix}40\\-39\\315\\-82\\94\end{matrix}\right]. 
 \end{align*}

Вариант N 19
Дана СЛАУ $AX = b$,
Проверить совместность по теореме Кронекера-Капелли. Если СЛАУ совместна, проверить единственность решения.
Для соответствующей однородной СЛАУ проверить существование нетривиального решения. В случае, если оно существует,
найти размерность пространства решений и составить ФСР и общее решение однородной  и неоднородной СЛАУ.


\begin{align*}
 A = \left[\begin{matrix}-7 & 5 & 0 & 9 & 0\\0 & -8 & 2 & 5 & -7\\-28 & -20 & 10 & 61 & -35\\-28 & 60 & -10 & 11 & 35\\2 & 1 & -6 & 7 & 6\end{matrix}\right],
\ b = \left[\begin{matrix}-25\\96\\380\\-580\\8\end{matrix}\right]. 
 \end{align*}

Вариант N 20
Дана СЛАУ $AX = b$,
Проверить совместность по теореме Кронекера-Капелли. Если СЛАУ совместна, проверить единственность решения.
Для соответствующей однородной СЛАУ проверить существование нетривиального решения. В случае, если оно существует,
найти размерность пространства решений и составить ФСР и общее решение однородной  и неоднородной СЛАУ.


\begin{align*}
 A = \left[\begin{matrix}-9 & -2 & 6 & -7 & 3\\0 & 3 & 5 & 4 & -7\\-36 & 4 & 44 & -12 & -16\\-36 & -20 & 4 & -44 & 40\\-9 & -2 & 6 & -2 & -9\end{matrix}\right],
\ b = \left[\begin{matrix}46\\-50\\-16\\384\\-11\end{matrix}\right]. 
 \end{align*}

Вариант N 21
Дана СЛАУ $AX = b$,
Проверить совместность по теореме Кронекера-Капелли. Если СЛАУ совместна, проверить единственность решения.
Для соответствующей однородной СЛАУ проверить существование нетривиального решения. В случае, если оно существует,
найти размерность пространства решений и составить ФСР и общее решение однородной  и неоднородной СЛАУ.


\begin{align*}
 A = \left[\begin{matrix}-3 & 2 & -2 & -8 & -8\\-25 & -30 & 14 & -24 & -48\\-109 & -114 & 50 & -120 & -216\\91 & 126 & -62 & 72 & 168\\282 & 372 & -180 & 240 & 528\end{matrix}\right],
\ b = \left[\begin{matrix}-60\\-20\\-260\\-100\\-120\end{matrix}\right]. 
 \end{align*}

Вариант N 22
Дана СЛАУ $AX = b$,
Проверить совместность по теореме Кронекера-Капелли. Если СЛАУ совместна, проверить единственность решения.
Для соответствующей однородной СЛАУ проверить существование нетривиального решения. В случае, если оно существует,
найти размерность пространства решений и составить ФСР и общее решение однородной  и неоднородной СЛАУ.


\begin{align*}
 A = \left[\begin{matrix}-1 & 8 & -2 & 2 & -9\\-3 & -6 & -5 & 6 & 5\\-15 & 0 & -26 & 30 & -7\\9 & 48 & 14 & -18 & -47\\-6 & 6 & -2 & 0 & 8\end{matrix}\right],
\ b = \left[\begin{matrix}2\\49\\202\\-190\\124\end{matrix}\right]. 
 \end{align*}

Вариант N 23
Дана СЛАУ $AX = b$,
Проверить совместность по теореме Кронекера-Капелли. Если СЛАУ совместна, проверить единственность решения.
Для соответствующей однородной СЛАУ проверить существование нетривиального решения. В случае, если оно существует,
найти размерность пространства решений и составить ФСР и общее решение однородной  и неоднородной СЛАУ.


\begin{align*}
 A = \left[\begin{matrix}4 & -2 & 3 & -6 & 1\\-29 & 32 & -13 & 6 & -31\\-129 & 152 & -53 & 6 & -151\\161 & -168 & 77 & -54 & 159\\467 & -496 & 219 & -138 & 473\end{matrix}\right],
\ b = \left[\begin{matrix}56\\-516\\-2356\\2804\\8188\end{matrix}\right]. 
 \end{align*}

Вариант N 24
Дана СЛАУ $AX = b$,
Проверить совместность по теореме Кронекера-Капелли. Если СЛАУ совместна, проверить единственность решения.
Для соответствующей однородной СЛАУ проверить существование нетривиального решения. В случае, если оно существует,
найти размерность пространства решений и составить ФСР и общее решение однородной  и неоднородной СЛАУ.


\begin{align*}
 A = \left[\begin{matrix}0 & 1 & 6 & -2 & 1\\16 & 3 & 2 & 30 & 3\\64 & 15 & 26 & 114 & 15\\-64 & -9 & 10 & -126 & -9\\-192 & -30 & 12 & -372 & -30\end{matrix}\right],
\ b = \left[\begin{matrix}56\\24\\264\\72\\48\end{matrix}\right]. 
 \end{align*}

Вариант N 25
Дана СЛАУ $AX = b$,
Проверить совместность по теореме Кронекера-Капелли. Если СЛАУ совместна, проверить единственность решения.
Для соответствующей однородной СЛАУ проверить существование нетривиального решения. В случае, если оно существует,
найти размерность пространства решений и составить ФСР и общее решение однородной  и неоднородной СЛАУ.


\begin{align*}
 A = \left[\begin{matrix}2 & 5 & 2 & 1 & 8\\8 & 5 & 9 & 8 & 1\\48 & 45 & 53 & 44 & 37\\-32 & -5 & -37 & -36 & 27\\6 & 0 & 1 & -5 & -4\end{matrix}\right],
\ b = \left[\begin{matrix}54\\15\\291\\141\\-54\end{matrix}\right]. 
 \end{align*}

Вариант N 26
Дана СЛАУ $AX = b$,
Проверить совместность по теореме Кронекера-Капелли. Если СЛАУ совместна, проверить единственность решения.
Для соответствующей однородной СЛАУ проверить существование нетривиального решения. В случае, если оно существует,
найти размерность пространства решений и составить ФСР и общее решение однородной  и неоднородной СЛАУ.


\begin{align*}
 A = \left[\begin{matrix}3 & 7 & -1 & -1 & -7\\28 & -8 & 32 & -36 & -4\\124 & -4 & 124 & -148 & -44\\-100 & 60 & -132 & 140 & -12\\-312 & 152 & -392 & 424 & -8\end{matrix}\right],
\ b = \left[\begin{matrix}-59\\-392\\-1804\\1332\\4232\end{matrix}\right]. 
 \end{align*}

Вариант N 27
Дана СЛАУ $AX = b$,
Проверить совместность по теореме Кронекера-Капелли. Если СЛАУ совместна, проверить единственность решения.
Для соответствующей однородной СЛАУ проверить существование нетривиального решения. В случае, если оно существует,
найти размерность пространства решений и составить ФСР и общее решение однородной  и неоднородной СЛАУ.


\begin{align*}
 A = \left[\begin{matrix}-7 & 9 & 6 & 7 & -3\\-6 & -7 & 6 & 2 & 5\\2 & 71 & -6 & 18 & -37\\0 & -7 & 3 & 6 & -2\\5 & 8 & 6 & 2 & 4\end{matrix}\right],
\ b = \left[\begin{matrix}-128\\7\\-547\\-35\\-34\end{matrix}\right]. 
 \end{align*}

Вариант N 28
Дана СЛАУ $AX = b$,
Проверить совместность по теореме Кронекера-Капелли. Если СЛАУ совместна, проверить единственность решения.
Для соответствующей однородной СЛАУ проверить существование нетривиального решения. В случае, если оно существует,
найти размерность пространства решений и составить ФСР и общее решение однородной  и неоднородной СЛАУ.


\begin{align*}
 A = \left[\begin{matrix}4 & 4 & 3 & 6 & -1\\6 & 9 & -8 & -7 & -9\\46 & 61 & -28 & -11 & -49\\-14 & -29 & 52 & 59 & 41\\6 & -5 & -8 & 8 & 7\end{matrix}\right],
\ b = \left[\begin{matrix}44\\64\\496\\-144\\26\end{matrix}\right]. 
 \end{align*}

Вариант N 29
Дана СЛАУ $AX = b$,
Проверить совместность по теореме Кронекера-Капелли. Если СЛАУ совместна, проверить единственность решения.
Для соответствующей однородной СЛАУ проверить существование нетривиального решения. В случае, если оно существует,
найти размерность пространства решений и составить ФСР и общее решение однородной  и неоднородной СЛАУ.


\begin{align*}
 A = \left[\begin{matrix}8 & 5 & -5 & 0 & 5\\-8 & -7 & -9 & 5 & -9\\-8 & -13 & -51 & 20 & -21\\56 & 43 & 21 & -20 & 51\\7 & -9 & 8 & 1 & 5\end{matrix}\right],
\ b = \left[\begin{matrix}79\\-4\\221\\253\\42\end{matrix}\right]. 
 \end{align*}

Вариант N 30
Дана СЛАУ $AX = b$,
Проверить совместность по теореме Кронекера-Капелли. Если СЛАУ совместна, проверить единственность решения.
Для соответствующей однородной СЛАУ проверить существование нетривиального решения. В случае, если оно существует,
найти размерность пространства решений и составить ФСР и общее решение однородной  и неоднородной СЛАУ.


\begin{align*}
 A = \left[\begin{matrix}1 & 1 & -8 & 3 & -2\\-3 & -2 & 9 & -6 & 2\\-11 & -6 & 13 & -18 & 2\\19 & 14 & -77 & 42 & -18\\4 & -5 & 4 & -6 & 1\end{matrix}\right],
\ b = \left[\begin{matrix}-99\\129\\249\\-1041\\117\end{matrix}\right]. 
 \end{align*}

Вариант N 31
Дана СЛАУ $AX = b$,
Проверить совместность по теореме Кронекера-Капелли. Если СЛАУ совместна, проверить единственность решения.
Для соответствующей однородной СЛАУ проверить существование нетривиального решения. В случае, если оно существует,
найти размерность пространства решений и составить ФСР и общее решение однородной  и неоднородной СЛАУ.


\begin{align*}
 A = \left[\begin{matrix}8 & -3 & 4 & -5 & 7\\42 & 28 & 31 & -45 & 63\\242 & 128 & 171 & -245 & 343\\-178 & -152 & -139 & 205 & -287\\-566 & -444 & -433 & 635 & -889\end{matrix}\right],
\ b = \left[\begin{matrix}9\\-124\\-584\\656\\1932\end{matrix}\right]. 
 \end{align*}

Вариант N 32
Дана СЛАУ $AX = b$,
Проверить совместность по теореме Кронекера-Капелли. Если СЛАУ совместна, проверить единственность решения.
Для соответствующей однородной СЛАУ проверить существование нетривиального решения. В случае, если оно существует,
найти размерность пространства решений и составить ФСР и общее решение однородной  и неоднородной СЛАУ.


\begin{align*}
 A = \left[\begin{matrix}-2 & 1 & -3 & -9 & 8\\5 & -2 & -2 & -3 & -1\\12 & -4 & -20 & -48 & 28\\-28 & 12 & -4 & -24 & 36\\3 & 9 & -8 & 7 & -5\end{matrix}\right],
\ b = \left[\begin{matrix}-54\\-20\\-296\\-136\\50\end{matrix}\right]. 
 \end{align*}

Вариант N 33
Дана СЛАУ $AX = b$,
Проверить совместность по теореме Кронекера-Капелли. Если СЛАУ совместна, проверить единственность решения.
Для соответствующей однородной СЛАУ проверить существование нетривиального решения. В случае, если оно существует,
найти размерность пространства решений и составить ФСР и общее решение однородной  и неоднородной СЛАУ.


\begin{align*}
 A = \left[\begin{matrix}-4 & 9 & 3 & -1 & 3\\-9 & 5 & -2 & 2 & -1\\24 & 7 & 17 & -11 & 13\\-1 & -3 & 1 & -2 & 3\\-2 & -6 & 0 & -3 & 8\end{matrix}\right],
\ b = \left[\begin{matrix}48\\62\\-104\\4\\-7\end{matrix}\right]. 
 \end{align*}

Вариант N 34
Дана СЛАУ $AX = b$,
Проверить совместность по теореме Кронекера-Капелли. Если СЛАУ совместна, проверить единственность решения.
Для соответствующей однородной СЛАУ проверить существование нетривиального решения. В случае, если оно существует,
найти размерность пространства решений и составить ФСР и общее решение однородной  и неоднородной СЛАУ.


\begin{align*}
 A = \left[\begin{matrix}7 & -2 & -7 & 3 & 6\\23 & -53 & -23 & 17 & -21\\143 & -273 & -143 & 97 & -81\\-87 & 257 & 87 & -73 & 129\\-289 & 779 & 289 & -231 & 363\end{matrix}\right],
\ b = \left[\begin{matrix}-79\\404\\1704\\-2336\\-6692\end{matrix}\right]. 
 \end{align*}

Вариант N 35
Дана СЛАУ $AX = b$,
Проверить совместность по теореме Кронекера-Капелли. Если СЛАУ совместна, проверить единственность решения.
Для соответствующей однородной СЛАУ проверить существование нетривиального решения. В случае, если оно существует,
найти размерность пространства решений и составить ФСР и общее решение однородной  и неоднородной СЛАУ.


\begin{align*}
 A = \left[\begin{matrix}-2 & -9 & 1 & 9 & 3\\14 & -72 & -12 & 17 & -11\\64 & -387 & -57 & 112 & -46\\-76 & 333 & 63 & -58 & 64\\-222 & 1026 & 186 & -201 & 183\end{matrix}\right],
\ b = \left[\begin{matrix}-75\\-440\\-2425\\1975\\6150\end{matrix}\right]. 
 \end{align*}

Вариант N 36
Дана СЛАУ $AX = b$,
Проверить совместность по теореме Кронекера-Капелли. Если СЛАУ совместна, проверить единственность решения.
Для соответствующей однородной СЛАУ проверить существование нетривиального решения. В случае, если оно существует,
найти размерность пространства решений и составить ФСР и общее решение однородной  и неоднородной СЛАУ.


\begin{align*}
 A = \left[\begin{matrix}-6 & 3 & -1 & -7 & 8\\1 & -1 & -9 & 6 & 7\\-22 & 13 & 33 & -45 & -4\\3 & 3 & -4 & -2 & -9\\7 & -4 & 7 & -8 & -3\end{matrix}\right],
\ b = \left[\begin{matrix}3\\22\\-79\\79\\28\end{matrix}\right]. 
 \end{align*}

Вариант N 37
Дана СЛАУ $AX = b$,
Проверить совместность по теореме Кронекера-Капелли. Если СЛАУ совместна, проверить единственность решения.
Для соответствующей однородной СЛАУ проверить существование нетривиального решения. В случае, если оно существует,
найти размерность пространства решений и составить ФСР и общее решение однородной  и неоднородной СЛАУ.


\begin{align*}
 A = \left[\begin{matrix}9 & 7 & 1 & 3 & 7\\-18 & 66 & 18 & -1 & 36\\-63 & 351 & 93 & 4 & 201\\117 & -309 & -87 & 14 & -159\\324 & -948 & -264 & 33 & -498\end{matrix}\right],
\ b = \left[\begin{matrix}-62\\-651\\-3441\\3069\\9393\end{matrix}\right]. 
 \end{align*}

Вариант N 38
Дана СЛАУ $AX = b$,
Проверить совместность по теореме Кронекера-Капелли. Если СЛАУ совместна, проверить единственность решения.
Для соответствующей однородной СЛАУ проверить существование нетривиального решения. В случае, если оно существует,
найти размерность пространства решений и составить ФСР и общее решение однородной  и неоднородной СЛАУ.


\begin{align*}
 A = \left[\begin{matrix}-6 & -2 & 7 & -1 & -2\\-2 & 1 & -6 & -1 & -4\\-14 & -13 & 58 & 1 & 12\\-8 & -2 & 8 & -1 & 5\\7 & -9 & 2 & -7 & 6\end{matrix}\right],
\ b = \left[\begin{matrix}-25\\-2\\-90\\-78\\-48\end{matrix}\right]. 
 \end{align*}

Вариант N 39
Дана СЛАУ $AX = b$,
Проверить совместность по теореме Кронекера-Капелли. Если СЛАУ совместна, проверить единственность решения.
Для соответствующей однородной СЛАУ проверить существование нетривиального решения. В случае, если оно существует,
найти размерность пространства решений и составить ФСР и общее решение однородной  и неоднородной СЛАУ.


\begin{align*}
 A = \left[\begin{matrix}-2 & 8 & 2 & 9 & 2\\20 & 52 & 40 & 32 & 20\\72 & 240 & 168 & 164 & 88\\-88 & -176 & -152 & -92 & -72\\-256 & -560 & -464 & -312 & -224\end{matrix}\right],
\ b = \left[\begin{matrix}-61\\-172\\-932\\444\\1576\end{matrix}\right]. 
 \end{align*}

Вариант N 40
Дана СЛАУ $AX = b$,
Проверить совместность по теореме Кронекера-Капелли. Если СЛАУ совместна, проверить единственность решения.
Для соответствующей однородной СЛАУ проверить существование нетривиального решения. В случае, если оно существует,
найти размерность пространства решений и составить ФСР и общее решение однородной  и неоднородной СЛАУ.


\begin{align*}
 A = \left[\begin{matrix}-1 & 5 & 5 & -3 & 7\\7 & -5 & 1 & 2 & 5\\-31 & 35 & 11 & -17 & 1\\5 & -3 & 4 & -7 & -4\\7 & -5 & 0 & -2 & -7\end{matrix}\right],
\ b = \left[\begin{matrix}29\\114\\-369\\-36\\-26\end{matrix}\right]. 
 \end{align*}

Вариант N 41
Дана СЛАУ $AX = b$,
Проверить совместность по теореме Кронекера-Капелли. Если СЛАУ совместна, проверить единственность решения.
Для соответствующей однородной СЛАУ проверить существование нетривиального решения. В случае, если оно существует,
найти размерность пространства решений и составить ФСР и общее решение однородной  и неоднородной СЛАУ.


\begin{align*}
 A = \left[\begin{matrix}2 & 4 & -8 & -8 & -2\\0 & 7 & 2 & 5 & -9\\8 & 51 & -22 & -7 & -53\\8 & -19 & -42 & -57 & 37\\4 & -9 & 7 & -9 & 1\end{matrix}\right],
\ b = \left[\begin{matrix}64\\-9\\211\\301\\131\end{matrix}\right]. 
 \end{align*}

Вариант N 42
Дана СЛАУ $AX = b$,
Проверить совместность по теореме Кронекера-Капелли. Если СЛАУ совместна, проверить единственность решения.
Для соответствующей однородной СЛАУ проверить существование нетривиального решения. В случае, если оно существует,
найти размерность пространства решений и составить ФСР и общее решение однородной  и неоднородной СЛАУ.


\begin{align*}
 A = \left[\begin{matrix}-6 & -6 & -9 & 8 & -6\\-4 & -9 & 9 & -6 & -6\\-2 & 18 & -63 & 48 & 6\\6 & 1 & -7 & -8 & 7\\0 & -8 & -9 & 1 & -3\end{matrix}\right],
\ b = \left[\begin{matrix}-49\\69\\-423\\35\\-23\end{matrix}\right]. 
 \end{align*}

Вариант N 43
Дана СЛАУ $AX = b$,
Проверить совместность по теореме Кронекера-Капелли. Если СЛАУ совместна, проверить единственность решения.
Для соответствующей однородной СЛАУ проверить существование нетривиального решения. В случае, если оно существует,
найти размерность пространства решений и составить ФСР и общее решение однородной  и неоднородной СЛАУ.


\begin{align*}
 A = \left[\begin{matrix}-5 & -2 & -6 & -2 & -8\\21 & -26 & 2 & -14 & 4\\69 & -110 & -10 & -62 & -8\\-99 & 98 & -26 & 50 & -40\\-282 & 300 & -60 & 156 & -96\end{matrix}\right],
\ b = \left[\begin{matrix}-132\\312\\852\\-1644\\-4536\end{matrix}\right]. 
 \end{align*}

Вариант N 44
Дана СЛАУ $AX = b$,
Проверить совместность по теореме Кронекера-Капелли. Если СЛАУ совместна, проверить единственность решения.
Для соответствующей однородной СЛАУ проверить существование нетривиального решения. В случае, если оно существует,
найти размерность пространства решений и составить ФСР и общее решение однородной  и неоднородной СЛАУ.


\begin{align*}
 A = \left[\begin{matrix}2 & -5 & -1 & 5 & 0\\-1 & 7 & -3 & -8 & -9\\2 & 13 & -15 & -17 & -36\\10 & -43 & 9 & 47 & 36\\5 & 8 & -2 & -2 & 8\end{matrix}\right],
\ b = \left[\begin{matrix}28\\-70\\-196\\364\\-27\end{matrix}\right]. 
 \end{align*}

Вариант N 45
Дана СЛАУ $AX = b$,
Проверить совместность по теореме Кронекера-Капелли. Если СЛАУ совместна, проверить единственность решения.
Для соответствующей однородной СЛАУ проверить существование нетривиального решения. В случае, если оно существует,
найти размерность пространства решений и составить ФСР и общее решение однородной  и неоднородной СЛАУ.


\begin{align*}
 A = \left[\begin{matrix}-3 & 0 & 7 & 8 & 1\\-5 & -8 & 0 & 3 & 8\\8 & 32 & 28 & 20 & -28\\-1 & -2 & 4 & 1 & 7\\-5 & 6 & 1 & -9 & -6\end{matrix}\right],
\ b = \left[\begin{matrix}-9\\-81\\288\\-27\\149\end{matrix}\right]. 
 \end{align*}

Вариант N 46
Дана СЛАУ $AX = b$,
Проверить совместность по теореме Кронекера-Капелли. Если СЛАУ совместна, проверить единственность решения.
Для соответствующей однородной СЛАУ проверить существование нетривиального решения. В случае, если оно существует,
найти размерность пространства решений и составить ФСР и общее решение однородной  и неоднородной СЛАУ.


\begin{align*}
 A = \left[\begin{matrix}4 & -1 & 2 & 4 & 8\\28 & 13 & -2 & 36 & 48\\124 & 49 & -2 & 156 & 216\\-100 & -55 & 14 & -132 & -168\\-312 & -162 & 36 & -408 & -528\end{matrix}\right],
\ b = \left[\begin{matrix}14\\146\\626\\-542\\-1668\end{matrix}\right]. 
 \end{align*}

Вариант N 47
Дана СЛАУ $AX = b$,
Проверить совместность по теореме Кронекера-Капелли. Если СЛАУ совместна, проверить единственность решения.
Для соответствующей однородной СЛАУ проверить существование нетривиального решения. В случае, если оно существует,
найти размерность пространства решений и составить ФСР и общее решение однородной  и неоднородной СЛАУ.


\begin{align*}
 A = \left[\begin{matrix}9 & 1 & -3 & -2 & 7\\-9 & 14 & 28 & -13 & 13\\-9 & 74 & 128 & -73 & 93\\81 & -66 & -152 & 57 & -37\\207 & -202 & -444 & 179 & -139\end{matrix}\right],
\ b = \left[\begin{matrix}47\\463\\2503\\-2127\\-6569\end{matrix}\right]. 
 \end{align*}

Вариант N 48
Дана СЛАУ $AX = b$,
Проверить совместность по теореме Кронекера-Капелли. Если СЛАУ совместна, проверить единственность решения.
Для соответствующей однородной СЛАУ проверить существование нетривиального решения. В случае, если оно существует,
найти размерность пространства решений и составить ФСР и общее решение однородной  и неоднородной СЛАУ.


\begin{align*}
 A = \left[\begin{matrix}-2 & -8 & 3 & -9 & -4\\-4 & 4 & -6 & -3 & -9\\14 & -44 & 39 & -12 & 33\\1 & 8 & 8 & -5 & 1\\9 & 3 & -9 & -9 & 6\end{matrix}\right],
\ b = \left[\begin{matrix}-38\\99\\-609\\62\\138\end{matrix}\right]. 
 \end{align*}

Вариант N 49
Дана СЛАУ $AX = b$,
Проверить совместность по теореме Кронекера-Капелли. Если СЛАУ совместна, проверить единственность решения.
Для соответствующей однородной СЛАУ проверить существование нетривиального решения. В случае, если оно существует,
найти размерность пространства решений и составить ФСР и общее решение однородной  и неоднородной СЛАУ.


\begin{align*}
 A = \left[\begin{matrix}8 & -2 & -2 & -6 & -8\\47 & 12 & -48 & 11 & -32\\267 & 52 & -248 & 31 & -192\\-203 & -68 & 232 & -79 & 128\\-641 & -196 & 704 & -213 & 416\end{matrix}\right],
\ b = \left[\begin{matrix}44\\-509\\-2369\\2721\\7987\end{matrix}\right]. 
 \end{align*}

Вариант N 50
Дана СЛАУ $AX = b$,
Проверить совместность по теореме Кронекера-Капелли. Если СЛАУ совместна, проверить единственность решения.
Для соответствующей однородной СЛАУ проверить существование нетривиального решения. В случае, если оно существует,
найти размерность пространства решений и составить ФСР и общее решение однородной  и неоднородной СЛАУ.


\begin{align*}
 A = \left[\begin{matrix}9 & 3 & -9 & 3 & -7\\-7 & -3 & 6 & 0 & -6\\55 & 21 & -51 & 9 & 3\\-9 & 8 & 5 & 9 & 1\\-2 & 2 & 3 & 8 & 9\end{matrix}\right],
\ b = \left[\begin{matrix}12\\-67\\304\\-18\\82\end{matrix}\right]. 
 \end{align*}

Вариант N 51
Дана СЛАУ $AX = b$,
Проверить совместность по теореме Кронекера-Капелли. Если СЛАУ совместна, проверить единственность решения.
Для соответствующей однородной СЛАУ проверить существование нетривиального решения. В случае, если оно существует,
найти размерность пространства решений и составить ФСР и общее решение однородной  и неоднородной СЛАУ.


\begin{align*}
 A = \left[\begin{matrix}6 & 6 & 0 & 5 & 3\\-2 & -5 & -3 & 4 & 7\\14 & -1 & -15 & 40 & 47\\34 & 49 & 15 & 0 & -23\\-5 & 6 & 6 & -9 & 6\end{matrix}\right],
\ b = \left[\begin{matrix}76\\71\\659\\-51\\-57\end{matrix}\right]. 
 \end{align*}

Вариант N 52
Дана СЛАУ $AX = b$,
Проверить совместность по теореме Кронекера-Капелли. Если СЛАУ совместна, проверить единственность решения.
Для соответствующей однородной СЛАУ проверить существование нетривиального решения. В случае, если оно существует,
найти размерность пространства решений и составить ФСР и общее решение однородной  и неоднородной СЛАУ.


\begin{align*}
 A = \left[\begin{matrix}-9 & -5 & 3 & 1 & -5\\-3 & -2 & -5 & 3 & 0\\-42 & -25 & -16 & 18 & -15\\-12 & -5 & 34 & -12 & -15\\3 & -5 & -8 & 8 & 2\end{matrix}\right],
\ b = \left[\begin{matrix}39\\19\\212\\22\\93\end{matrix}\right]. 
 \end{align*}

Вариант N 53
Дана СЛАУ $AX = b$,
Проверить совместность по теореме Кронекера-Капелли. Если СЛАУ совместна, проверить единственность решения.
Для соответствующей однородной СЛАУ проверить существование нетривиального решения. В случае, если оно существует,
найти размерность пространства решений и составить ФСР и общее решение однородной  и неоднородной СЛАУ.


\begin{align*}
 A = \left[\begin{matrix}-5 & -2 & -5 & 4 & 7\\9 & -5 & 1 & 7 & -1\\21 & -26 & -11 & 40 & 17\\-51 & 14 & -19 & -16 & 25\\4 & 6 & -1 & 0 & 4\end{matrix}\right],
\ b = \left[\begin{matrix}-68\\48\\-12\\-396\\-12\end{matrix}\right]. 
 \end{align*}

Вариант N 54
Дана СЛАУ $AX = b$,
Проверить совместность по теореме Кронекера-Капелли. Если СЛАУ совместна, проверить единственность решения.
Для соответствующей однородной СЛАУ проверить существование нетривиального решения. В случае, если оно существует,
найти размерность пространства решений и составить ФСР и общее решение однородной  и неоднородной СЛАУ.


\begin{align*}
 A = \left[\begin{matrix}0 & 0 & -8 & -2 & -7\\-6 & 6 & 4 & -2 & 4\\-30 & 30 & -12 & -18 & -8\\30 & -30 & -52 & 2 & -48\\-3 & 4 & -2 & 8 & -6\end{matrix}\right],
\ b = \left[\begin{matrix}-5\\-22\\-130\\90\\-32\end{matrix}\right]. 
 \end{align*}

Вариант N 55
Дана СЛАУ $AX = b$,
Проверить совместность по теореме Кронекера-Капелли. Если СЛАУ совместна, проверить единственность решения.
Для соответствующей однородной СЛАУ проверить существование нетривиального решения. В случае, если оно существует,
найти размерность пространства решений и составить ФСР и общее решение однородной  и неоднородной СЛАУ.


\begin{align*}
 A = \left[\begin{matrix}8 & -3 & 7 & -8 & -9\\-5 & 6 & 4 & 4 & 7\\-1 & 21 & 41 & -4 & 8\\49 & -39 & 1 & -44 & -62\\8 & 4 & -3 & -4 & 7\end{matrix}\right],
\ b = \left[\begin{matrix}96\\-129\\-357\\933\\8\end{matrix}\right]. 
 \end{align*}

Вариант N 56
Дана СЛАУ $AX = b$,
Проверить совместность по теореме Кронекера-Капелли. Если СЛАУ совместна, проверить единственность решения.
Для соответствующей однородной СЛАУ проверить существование нетривиального решения. В случае, если оно существует,
найти размерность пространства решений и составить ФСР и общее решение однородной  и неоднородной СЛАУ.


\begin{align*}
 A = \left[\begin{matrix}-3 & 4 & 5 & -7 & -3\\-2 & -8 & -8 & 3 & -1\\-17 & -20 & -17 & -9 & -13\\-1 & 44 & 47 & -33 & -5\\4 & -1 & 9 & 6 & 1\end{matrix}\right],
\ b = \left[\begin{matrix}-63\\-13\\-241\\-137\\33\end{matrix}\right]. 
 \end{align*}

Вариант N 57
Дана СЛАУ $AX = b$,
Проверить совместность по теореме Кронекера-Капелли. Если СЛАУ совместна, проверить единственность решения.
Для соответствующей однородной СЛАУ проверить существование нетривиального решения. В случае, если оно существует,
найти размерность пространства решений и составить ФСР и общее решение однородной  и неоднородной СЛАУ.


\begin{align*}
 A = \left[\begin{matrix}8 & 8 & -9 & 6 & -4\\7 & 77 & -76 & 24 & 19\\67 & 417 & -416 & 144 & 79\\-3 & -353 & 344 & -96 & -111\\-41 & -1091 & 1068 & -312 & -317\end{matrix}\right],
\ b = \left[\begin{matrix}21\\219\\1179\\-1011\\-3117\end{matrix}\right]. 
 \end{align*}

Вариант N 58
Дана СЛАУ $AX = b$,
Проверить совместность по теореме Кронекера-Капелли. Если СЛАУ совместна, проверить единственность решения.
Для соответствующей однородной СЛАУ проверить существование нетривиального решения. В случае, если оно существует,
найти размерность пространства решений и составить ФСР и общее решение однородной  и неоднородной СЛАУ.


\begin{align*}
 A = \left[\begin{matrix}-3 & -4 & -6 & -1 & 8\\-6 & 5 & -6 & -7 & 8\\18 & -41 & 6 & 31 & -8\\-4 & 0 & 3 & 5 & -6\\7 & 6 & -7 & 4 & -2\end{matrix}\right],
\ b = \left[\begin{matrix}-69\\-78\\114\\-26\\-57\end{matrix}\right]. 
 \end{align*}

Вариант N 59
Дана СЛАУ $AX = b$,
Проверить совместность по теореме Кронекера-Капелли. Если СЛАУ совместна, проверить единственность решения.
Для соответствующей однородной СЛАУ проверить существование нетривиального решения. В случае, если оно существует,
найти размерность пространства решений и составить ФСР и общее решение однородной  и неоднородной СЛАУ.


\begin{align*}
 A = \left[\begin{matrix}8 & -3 & -8 & -6 & -6\\19 & -39 & 6 & -63 & -13\\119 & -204 & 6 & -333 & -83\\-71 & 186 & -54 & 297 & 47\\-237 & 567 & -138 & 909 & 159\end{matrix}\right],
\ b = \left[\begin{matrix}79\\482\\2647\\-2173\\-6756\end{matrix}\right]. 
 \end{align*}

Вариант N 60
Дана СЛАУ $AX = b$,
Проверить совместность по теореме Кронекера-Капелли. Если СЛАУ совместна, проверить единственность решения.
Для соответствующей однородной СЛАУ проверить существование нетривиального решения. В случае, если оно существует,
найти размерность пространства решений и составить ФСР и общее решение однородной  и неоднородной СЛАУ.


\begin{align*}
 A = \left[\begin{matrix}8 & 9 & -3 & 0 & -2\\3 & 9 & 0 & -4 & 6\\44 & 72 & -12 & -16 & 16\\20 & 0 & -12 & 16 & -32\\3 & -9 & 8 & -6 & -6\end{matrix}\right],
\ b = \left[\begin{matrix}-51\\-22\\-292\\-116\\-28\end{matrix}\right]. 
 \end{align*}

Вариант N 61
Дана СЛАУ $AX = b$,
Проверить совместность по теореме Кронекера-Капелли. Если СЛАУ совместна, проверить единственность решения.
Для соответствующей однородной СЛАУ проверить существование нетривиального решения. В случае, если оно существует,
найти размерность пространства решений и составить ФСР и общее решение однородной  и неоднородной СЛАУ.


\begin{align*}
 A = \left[\begin{matrix}8 & -9 & 4 & -2 & 7\\3 & 6 & -7 & 8 & -9\\12 & -51 & 40 & -38 & 57\\7 & -2 & -9 & 6 & 2\\7 & -5 & 3 & 9 & -7\end{matrix}\right],
\ b = \left[\begin{matrix}116\\-103\\760\\-31\\8\end{matrix}\right]. 
 \end{align*}

Вариант N 62
Дана СЛАУ $AX = b$,
Проверить совместность по теореме Кронекера-Капелли. Если СЛАУ совместна, проверить единственность решения.
Для соответствующей однородной СЛАУ проверить существование нетривиального решения. В случае, если оно существует,
найти размерность пространства решений и составить ФСР и общее решение однородной  и неоднородной СЛАУ.


\begin{align*}
 A = \left[\begin{matrix}5 & 0 & -7 & -2 & 4\\-3 & -2 & 4 & 3 & 2\\35 & 10 & -48 & -23 & 6\\-6 & -4 & 2 & -9 & -5\\4 & 0 & -5 & 0 & -4\end{matrix}\right],
\ b = \left[\begin{matrix}9\\-46\\266\\73\\6\end{matrix}\right]. 
 \end{align*}

Вариант N 63
Дана СЛАУ $AX = b$,
Проверить совместность по теореме Кронекера-Капелли. Если СЛАУ совместна, проверить единственность решения.
Для соответствующей однородной СЛАУ проверить существование нетривиального решения. В случае, если оно существует,
найти размерность пространства решений и составить ФСР и общее решение однородной  и неоднородной СЛАУ.


\begin{align*}
 A = \left[\begin{matrix}2 & -3 & -6 & -4 & -1\\-4 & -7 & -1 & 7 & 5\\-10 & -37 & -22 & 16 & 17\\22 & 19 & -14 & -40 & -23\\-7 & 7 & 7 & -3 & -6\end{matrix}\right],
\ b = \left[\begin{matrix}60\\41\\344\\16\\-107\end{matrix}\right]. 
 \end{align*}

Вариант N 64
Дана СЛАУ $AX = b$,
Проверить совместность по теореме Кронекера-Капелли. Если СЛАУ совместна, проверить единственность решения.
Для соответствующей однородной СЛАУ проверить существование нетривиального решения. В случае, если оно существует,
найти размерность пространства решений и составить ФСР и общее решение однородной  и неоднородной СЛАУ.


\begin{align*}
 A = \left[\begin{matrix}6 & 9 & 4 & 5 & 4\\9 & 1 & 6 & -8 & -4\\-27 & 22 & -18 & 55 & 32\\-1 & 5 & 8 & 6 & 3\\3 & 2 & 7 & 8 & -7\end{matrix}\right],
\ b = \left[\begin{matrix}-5\\114\\-585\\-51\\-40\end{matrix}\right]. 
 \end{align*}

Вариант N 65
Дана СЛАУ $AX = b$,
Проверить совместность по теореме Кронекера-Капелли. Если СЛАУ совместна, проверить единственность решения.
Для соответствующей однородной СЛАУ проверить существование нетривиального решения. В случае, если оно существует,
найти размерность пространства решений и составить ФСР и общее решение однородной  и неоднородной СЛАУ.


\begin{align*}
 A = \left[\begin{matrix}5 & 7 & 4 & 1 & -4\\-10 & -24 & 42 & 43 & 3\\-35 & -99 & 222 & 218 & 3\\65 & 141 & -198 & -212 & -27\\180 & 402 & -606 & -639 & -69\end{matrix}\right],
\ b = \left[\begin{matrix}-39\\-17\\-202\\-32\\21\end{matrix}\right]. 
 \end{align*}

Вариант N 66
Дана СЛАУ $AX = b$,
Проверить совместность по теореме Кронекера-Капелли. Если СЛАУ совместна, проверить единственность решения.
Для соответствующей однородной СЛАУ проверить существование нетривиального решения. В случае, если оно существует,
найти размерность пространства решений и составить ФСР и общее решение однородной  и неоднородной СЛАУ.


\begin{align*}
 A = \left[\begin{matrix}7 & -1 & 6 & 5 & -5\\2 & -3 & 3 & 4 & -6\\31 & -18 & 33 & 35 & -45\\11 & 12 & 3 & -5 & 15\\1 & 2 & 1 & 7 & 3\end{matrix}\right],
\ b = \left[\begin{matrix}58\\38\\364\\-16\\44\end{matrix}\right]. 
 \end{align*}

Вариант N 67
Дана СЛАУ $AX = b$,
Проверить совместность по теореме Кронекера-Капелли. Если СЛАУ совместна, проверить единственность решения.
Для соответствующей однородной СЛАУ проверить существование нетривиального решения. В случае, если оно существует,
найти размерность пространства решений и составить ФСР и общее решение однородной  и неоднородной СЛАУ.


\begin{align*}
 A = \left[\begin{matrix}-2 & -7 & 2 & -3 & -1\\-18 & -53 & 18 & -22 & -29\\-98 & -293 & 98 & -122 & -149\\82 & 237 & -82 & 98 & 141\\254 & 739 & -254 & 306 & 427\end{matrix}\right],
\ b = \left[\begin{matrix}60\\670\\3590\\-3110\\-9570\end{matrix}\right]. 
 \end{align*}

Вариант N 68
Дана СЛАУ $AX = b$,
Проверить совместность по теореме Кронекера-Капелли. Если СЛАУ совместна, проверить единственность решения.
Для соответствующей однородной СЛАУ проверить существование нетривиального решения. В случае, если оно существует,
найти размерность пространства решений и составить ФСР и общее решение однородной  и неоднородной СЛАУ.


\begin{align*}
 A = \left[\begin{matrix}4 & 2 & 3 & -7 & -3\\6 & 38 & -3 & -43 & -42\\46 & 198 & -3 & -243 & -222\\-14 & -182 & 27 & 187 & 198\\-58 & -554 & 69 & 589 & 606\end{matrix}\right],
\ b = \left[\begin{matrix}-113\\-982\\-5362\\4458\\13826\end{matrix}\right]. 
 \end{align*}

Вариант N 69
Дана СЛАУ $AX = b$,
Проверить совместность по теореме Кронекера-Капелли. Если СЛАУ совместна, проверить единственность решения.
Для соответствующей однородной СЛАУ проверить существование нетривиального решения. В случае, если оно существует,
найти размерность пространства решений и составить ФСР и общее решение однородной  и неоднородной СЛАУ.


\begin{align*}
 A = \left[\begin{matrix}9 & 3 & 3 & -5 & -2\\17 & -36 & 19 & -60 & -26\\112 & -171 & 104 & -315 & -136\\-58 & 189 & -86 & 285 & 124\\-201 & 558 & -267 & 870 & 378\end{matrix}\right],
\ b = \left[\begin{matrix}0\\10\\50\\-50\\-150\end{matrix}\right]. 
 \end{align*}

Вариант N 70
Дана СЛАУ $AX = b$,
Проверить совместность по теореме Кронекера-Капелли. Если СЛАУ совместна, проверить единственность решения.
Для соответствующей однородной СЛАУ проверить существование нетривиального решения. В случае, если оно существует,
найти размерность пространства решений и составить ФСР и общее решение однородной  и неоднородной СЛАУ.


\begin{align*}
 A = \left[\begin{matrix}-2 & 3 & -1 & -3 & 9\\-7 & -9 & -4 & -1 & 6\\-43 & -33 & -24 & -17 & 66\\27 & 57 & 16 & -7 & 6\\0 & -9 & 5 & -1 & 6\end{matrix}\right],
\ b = \left[\begin{matrix}52\\6\\238\\178\\14\end{matrix}\right]. 
 \end{align*}

Вариант N 71
Дана СЛАУ $AX = b$,
Проверить совместность по теореме Кронекера-Капелли. Если СЛАУ совместна, проверить единственность решения.
Для соответствующей однородной СЛАУ проверить существование нетривиального решения. В случае, если оно существует,
найти размерность пространства решений и составить ФСР и общее решение однородной  и неоднородной СЛАУ.


\begin{align*}
 A = \left[\begin{matrix}-2 & -2 & 8 & 2 & -8\\3 & -6 & -9 & 5 & 9\\-18 & 18 & 60 & -14 & -60\\7 & -6 & -8 & 0 & -2\\0 & 6 & -3 & 5 & 0\end{matrix}\right],
\ b = \left[\begin{matrix}-98\\147\\-882\\77\\18\end{matrix}\right]. 
 \end{align*}

Вариант N 72
Дана СЛАУ $AX = b$,
Проверить совместность по теореме Кронекера-Капелли. Если СЛАУ совместна, проверить единственность решения.
Для соответствующей однородной СЛАУ проверить существование нетривиального решения. В случае, если оно существует,
найти размерность пространства решений и составить ФСР и общее решение однородной  и неоднородной СЛАУ.


\begin{align*}
 A = \left[\begin{matrix}6 & 6 & 8 & -7 & 4\\-17 & 13 & 59 & -11 & 27\\-67 & 83 & 319 & -76 & 147\\103 & -47 & -271 & 34 & -123\\291 & -159 & -837 & 123 & -381\end{matrix}\right],
\ b = \left[\begin{matrix}41\\-77\\-262\\508\\1401\end{matrix}\right]. 
 \end{align*}

Вариант N 73
Дана СЛАУ $AX = b$,
Проверить совместность по теореме Кронекера-Капелли. Если СЛАУ совместна, проверить единственность решения.
Для соответствующей однородной СЛАУ проверить существование нетривиального решения. В случае, если оно существует,
найти размерность пространства решений и составить ФСР и общее решение однородной  и неоднородной СЛАУ.


\begin{align*}
 A = \left[\begin{matrix}7 & 9 & -8 & 9 & -3\\-7 & 36 & -22 & 76 & -52\\-7 & 216 & -142 & 416 & -272\\63 & -144 & 78 & -344 & 248\\161 & -468 & 266 & -1068 & 756\end{matrix}\right],
\ b = \left[\begin{matrix}0\\90\\450\\-450\\-1350\end{matrix}\right]. 
 \end{align*}

Вариант N 74
Дана СЛАУ $AX = b$,
Проверить совместность по теореме Кронекера-Капелли. Если СЛАУ совместна, проверить единственность решения.
Для соответствующей однородной СЛАУ проверить существование нетривиального решения. В случае, если оно существует,
найти размерность пространства решений и составить ФСР и общее решение однородной  и неоднородной СЛАУ.


\begin{align*}
 A = \left[\begin{matrix}5 & -7 & 9 & -6 & 0\\4 & -9 & 4 & 8 & -4\\4 & 8 & 20 & -56 & 16\\-3 & 8 & -5 & 0 & -2\\-1 & -5 & 0 & 3 & 4\end{matrix}\right],
\ b = \left[\begin{matrix}90\\25\\260\\-45\\-3\end{matrix}\right]. 
 \end{align*}

Вариант N 75
Дана СЛАУ $AX = b$,
Проверить совместность по теореме Кронекера-Капелли. Если СЛАУ совместна, проверить единственность решения.
Для соответствующей однородной СЛАУ проверить существование нетривиального решения. В случае, если оно существует,
найти размерность пространства решений и составить ФСР и общее решение однородной  и неоднородной СЛАУ.


\begin{align*}
 A = \left[\begin{matrix}7 & -6 & -2 & 2 & 7\\8 & -8 & -6 & 9 & -9\\61 & -58 & -36 & 51 & -24\\-19 & 22 & 24 & -39 & 66\\-4 & 6 & 3 & 9 & -7\end{matrix}\right],
\ b = \left[\begin{matrix}-112\\29\\-191\\-481\\161\end{matrix}\right]. 
 \end{align*}

Вариант N 76
Дана СЛАУ $AX = b$,
Проверить совместность по теореме Кронекера-Капелли. Если СЛАУ совместна, проверить единственность решения.
Для соответствующей однородной СЛАУ проверить существование нетривиального решения. В случае, если оно существует,
найти размерность пространства решений и составить ФСР и общее решение однородной  и неоднородной СЛАУ.


\begin{align*}
 A = \left[\begin{matrix}5 & -8 & -9 & -2 & -2\\45 & -4 & -32 & -1 & -16\\240 & -44 & -187 & -11 & -86\\-210 & -4 & 133 & -1 & 74\\-645 & 12 & 426 & 3 & 228\end{matrix}\right],
\ b = \left[\begin{matrix}-140\\-510\\-2970\\2130\\6810\end{matrix}\right]. 
 \end{align*}

Вариант N 77
Дана СЛАУ $AX = b$,
Проверить совместность по теореме Кронекера-Капелли. Если СЛАУ совместна, проверить единственность решения.
Для соответствующей однородной СЛАУ проверить существование нетривиального решения. В случае, если оно существует,
найти размерность пространства решений и составить ФСР и общее решение однородной  и неоднородной СЛАУ.


\begin{align*}
 A = \left[\begin{matrix}-2 & 0 & -8 & 6 & -6\\7 & -2 & -2 & -9 & 3\\-43 & 10 & -22 & 69 & -39\\5 & -7 & 9 & 6 & -2\\-3 & 7 & 2 & -5 & -5\end{matrix}\right],
\ b = \left[\begin{matrix}-28\\34\\-282\\-10\\-28\end{matrix}\right]. 
 \end{align*}

Вариант N 78
Дана СЛАУ $AX = b$,
Проверить совместность по теореме Кронекера-Капелли. Если СЛАУ совместна, проверить единственность решения.
Для соответствующей однородной СЛАУ проверить существование нетривиального решения. В случае, если оно существует,
найти размерность пространства решений и составить ФСР и общее решение однородной  и неоднородной СЛАУ.


\begin{align*}
 A = \left[\begin{matrix}2 & 5 & -2 & 5 & -8\\-5 & 2 & -9 & 0 & -9\\-17 & 30 & -53 & 20 & -77\\33 & 10 & 37 & 20 & 13\\-9 & -6 & -5 & 0 & -9\end{matrix}\right],
\ b = \left[\begin{matrix}-27\\-88\\-548\\332\\-60\end{matrix}\right]. 
 \end{align*}

Вариант N 79
Дана СЛАУ $AX = b$,
Проверить совместность по теореме Кронекера-Капелли. Если СЛАУ совместна, проверить единственность решения.
Для соответствующей однородной СЛАУ проверить существование нетривиального решения. В случае, если оно существует,
найти размерность пространства решений и составить ФСР и общее решение однородной  и неоднородной СЛАУ.


\begin{align*}
 A = \left[\begin{matrix}-8 & -4 & -2 & 1 & 0\\-3 & 8 & 6 & -7 & -7\\-44 & 16 & 16 & -24 & -28\\-20 & -48 & -32 & 32 & 28\\-5 & 4 & 2 & -6 & 0\end{matrix}\right],
\ b = \left[\begin{matrix}-54\\84\\120\\-552\\55\end{matrix}\right]. 
 \end{align*}

Вариант N 80
Дана СЛАУ $AX = b$,
Проверить совместность по теореме Кронекера-Капелли. Если СЛАУ совместна, проверить единственность решения.
Для соответствующей однородной СЛАУ проверить существование нетривиального решения. В случае, если оно существует,
найти размерность пространства решений и составить ФСР и общее решение однородной  и неоднородной СЛАУ.


\begin{align*}
 A = \left[\begin{matrix}5 & -7 & 4 & -2 & 0\\-7 & 0 & -1 & -4 & -9\\-13 & -21 & 8 & -22 & -36\\43 & -21 & 16 & 10 & 36\\3 & 5 & -1 & -8 & 1\end{matrix}\right],
\ b = \left[\begin{matrix}-55\\71\\119\\-449\\-105\end{matrix}\right]. 
 \end{align*}

Вариант N 81
Дана СЛАУ $AX = b$,
Проверить совместность по теореме Кронекера-Капелли. Если СЛАУ совместна, проверить единственность решения.
Для соответствующей однородной СЛАУ проверить существование нетривиального решения. В случае, если оно существует,
найти размерность пространства решений и составить ФСР и общее решение однородной  и неоднородной СЛАУ.


\begin{align*}
 A = \left[\begin{matrix}1 & 4 & -7 & -9 & -6\\-8 & 8 & 8 & -2 & -9\\35 & -20 & -53 & -19 & 18\\7 & 2 & -7 & 6 & -6\\-4 & 2 & 1 & -1 & -7\end{matrix}\right],
\ b = \left[\begin{matrix}34\\-17\\170\\43\\43\end{matrix}\right]. 
 \end{align*}

Вариант N 82
Дана СЛАУ $AX = b$,
Проверить совместность по теореме Кронекера-Капелли. Если СЛАУ совместна, проверить единственность решения.
Для соответствующей однородной СЛАУ проверить существование нетривиального решения. В случае, если оно существует,
найти размерность пространства решений и составить ФСР и общее решение однородной  и неоднородной СЛАУ.


\begin{align*}
 A = \left[\begin{matrix}-1 & 9 & -6 & 4 & 7\\5 & -7 & -3 & 8 & -2\\22 & -8 & -33 & 52 & 11\\-28 & 62 & -3 & -28 & 31\\-2 & 5 & 4 & 5 & -4\end{matrix}\right],
\ b = \left[\begin{matrix}-66\\-74\\-568\\172\\3\end{matrix}\right]. 
 \end{align*}

Вариант N 83
Дана СЛАУ $AX = b$,
Проверить совместность по теореме Кронекера-Капелли. Если СЛАУ совместна, проверить единственность решения.
Для соответствующей однородной СЛАУ проверить существование нетривиального решения. В случае, если оно существует,
найти размерность пространства решений и составить ФСР и общее решение однородной  и неоднородной СЛАУ.


\begin{align*}
 A = \left[\begin{matrix}3 & 4 & -8 & 6 & 1\\34 & 42 & 1 & 28 & 38\\179 & 222 & -19 & 158 & 193\\-161 & -198 & -29 & -122 & -187\\-492 & -606 & -63 & -384 & -564\end{matrix}\right],
\ b = \left[\begin{matrix}45\\-5\\110\\160\\345\end{matrix}\right]. 
 \end{align*}

Вариант N 84
Дана СЛАУ $AX = b$,
Проверить совместность по теореме Кронекера-Капелли. Если СЛАУ совместна, проверить единственность решения.
Для соответствующей однородной СЛАУ проверить существование нетривиального решения. В случае, если оно существует,
найти размерность пространства решений и составить ФСР и общее решение однородной  и неоднородной СЛАУ.


\begin{align*}
 A = \left[\begin{matrix}-5 & 3 & 1 & 3 & -6\\7 & 8 & 4 & -5 & 7\\20 & 49 & 23 & -16 & 17\\-50 & -31 & -17 & 34 & -53\\4 & -8 & -7 & 4 & 1\end{matrix}\right],
\ b = \left[\begin{matrix}-45\\163\\680\\-950\\-41\end{matrix}\right]. 
 \end{align*}

Вариант N 85
Дана СЛАУ $AX = b$,
Проверить совместность по теореме Кронекера-Капелли. Если СЛАУ совместна, проверить единственность решения.
Для соответствующей однородной СЛАУ проверить существование нетривиального решения. В случае, если оно существует,
найти размерность пространства решений и составить ФСР и общее решение однородной  и неоднородной СЛАУ.


\begin{align*}
 A = \left[\begin{matrix}-7 & 9 & -8 & -8 & -2\\-36 & 48 & -12 & -48 & -24\\-172 & 228 & -80 & -224 & -104\\116 & -156 & 16 & 160 & 88\\376 & -504 & 80 & 512 & 272\end{matrix}\right],
\ b = \left[\begin{matrix}-48\\-156\\-816\\432\\1488\end{matrix}\right]. 
 \end{align*}

Вариант N 86
Дана СЛАУ $AX = b$,
Проверить совместность по теореме Кронекера-Капелли. Если СЛАУ совместна, проверить единственность решения.
Для соответствующей однородной СЛАУ проверить существование нетривиального решения. В случае, если оно существует,
найти размерность пространства решений и составить ФСР и общее решение однородной  и неоднородной СЛАУ.


\begin{align*}
 A = \left[\begin{matrix}-4 & 6 & 5 & -8 & 7\\-3 & -8 & -2 & 1 & 6\\-1 & 64 & 30 & -37 & -2\\-5 & -8 & 7 & 1 & -3\\-7 & -2 & 2 & -6 & 0\end{matrix}\right],
\ b = \left[\begin{matrix}-88\\-40\\-152\\60\\-14\end{matrix}\right]. 
 \end{align*}

Вариант N 87
Дана СЛАУ $AX = b$,
Проверить совместность по теореме Кронекера-Капелли. Если СЛАУ совместна, проверить единственность решения.
Для соответствующей однородной СЛАУ проверить существование нетривиального решения. В случае, если оно существует,
найти размерность пространства решений и составить ФСР и общее решение однородной  и неоднородной СЛАУ.


\begin{align*}
 A = \left[\begin{matrix}-8 & 0 & -9 & 1 & 2\\-60 & -8 & -11 & -13 & -14\\-264 & -32 & -71 & -49 & -50\\216 & 32 & 17 & 55 & 62\\672 & 96 & 78 & 162 & 180\end{matrix}\right],
\ b = \left[\begin{matrix}23\\309\\1305\\-1167\\-3570\end{matrix}\right]. 
 \end{align*}

Вариант N 88
Дана СЛАУ $AX = b$,
Проверить совместность по теореме Кронекера-Капелли. Если СЛАУ совместна, проверить единственность решения.
Для соответствующей однородной СЛАУ проверить существование нетривиального решения. В случае, если оно существует,
найти размерность пространства решений и составить ФСР и общее решение однородной  и неоднородной СЛАУ.


\begin{align*}
 A = \left[\begin{matrix}6 & 9 & -6 & -6 & -3\\3 & 5 & -9 & -5 & 7\\30 & 47 & -54 & -38 & 19\\6 & 7 & 18 & 2 & -37\\-4 & -6 & -4 & -6 & -6\end{matrix}\right],
\ b = \left[\begin{matrix}54\\31\\286\\38\\0\end{matrix}\right]. 
 \end{align*}

Вариант N 89
Дана СЛАУ $AX = b$,
Проверить совместность по теореме Кронекера-Капелли. Если СЛАУ совместна, проверить единственность решения.
Для соответствующей однородной СЛАУ проверить существование нетривиального решения. В случае, если оно существует,
найти размерность пространства решений и составить ФСР и общее решение однородной  и неоднородной СЛАУ.


\begin{align*}
 A = \left[\begin{matrix}7 & 9 & 6 & 9 & 7\\-1 & -7 & 9 & 7 & -8\\24 & 8 & 60 & 64 & -4\\32 & 64 & -12 & 8 & 60\\-2 & 7 & 6 & 4 & -5\end{matrix}\right],
\ b = \left[\begin{matrix}42\\-128\\-344\\680\\20\end{matrix}\right]. 
 \end{align*}

Вариант N 90
Дана СЛАУ $AX = b$,
Проверить совместность по теореме Кронекера-Капелли. Если СЛАУ совместна, проверить единственность решения.
Для соответствующей однородной СЛАУ проверить существование нетривиального решения. В случае, если оно существует,
найти размерность пространства решений и составить ФСР и общее решение однородной  и неоднородной СЛАУ.


\begin{align*}
 A = \left[\begin{matrix}2 & 0 & -1 & -1 & -4\\5 & 1 & 0 & 0 & -1\\33 & 5 & -4 & -4 & -21\\-17 & -5 & -4 & -4 & -11\\-9 & 4 & 0 & -4 & -4\end{matrix}\right],
\ b = \left[\begin{matrix}28\\37\\297\\-73\\-72\end{matrix}\right]. 
 \end{align*}

Вариант N 91
Дана СЛАУ $AX = b$,
Проверить совместность по теореме Кронекера-Капелли. Если СЛАУ совместна, проверить единственность решения.
Для соответствующей однородной СЛАУ проверить существование нетривиального решения. В случае, если оно существует,
найти размерность пространства решений и составить ФСР и общее решение однородной  и неоднородной СЛАУ.


\begin{align*}
 A = \left[\begin{matrix}2 & 6 & 7 & -7 & -5\\7 & -4 & -6 & -4 & -2\\-29 & 38 & 51 & -1 & -5\\-6 & -7 & -8 & 2 & 6\\5 & -8 & 6 & -1 & -5\end{matrix}\right],
\ b = \left[\begin{matrix}22\\13\\1\\-50\\-38\end{matrix}\right]. 
 \end{align*}

Вариант N 92
Дана СЛАУ $AX = b$,
Проверить совместность по теореме Кронекера-Капелли. Если СЛАУ совместна, проверить единственность решения.
Для соответствующей однородной СЛАУ проверить существование нетривиального решения. В случае, если оно существует,
найти размерность пространства решений и составить ФСР и общее решение однородной  и неоднородной СЛАУ.


\begin{align*}
 A = \left[\begin{matrix}1 & 1 & -1 & 2 & 8\\2 & 8 & 9 & 3 & 3\\-6 & -36 & -49 & -7 & 17\\3 & -9 & 3 & 5 & -1\\-3 & -7 & 2 & 3 & -1\end{matrix}\right],
\ b = \left[\begin{matrix}7\\118\\-562\\-31\\-19\end{matrix}\right]. 
 \end{align*}

Вариант N 93
Дана СЛАУ $AX = b$,
Проверить совместность по теореме Кронекера-Капелли. Если СЛАУ совместна, проверить единственность решения.
Для соответствующей однородной СЛАУ проверить существование нетривиального решения. В случае, если оно существует,
найти размерность пространства решений и составить ФСР и общее решение однородной  и неоднородной СЛАУ.


\begin{align*}
 A = \left[\begin{matrix}3 & -8 & 6 & 2 & 9\\2 & -9 & 6 & 6 & 4\\17 & -60 & 42 & 30 & 43\\1 & 12 & -6 & -18 & 11\\9 & -8 & 7 & -9 & 5\end{matrix}\right],
\ b = \left[\begin{matrix}150\\144\\1026\\-126\\113\end{matrix}\right]. 
 \end{align*}

Вариант N 94
Дана СЛАУ $AX = b$,
Проверить совместность по теореме Кронекера-Капелли. Если СЛАУ совместна, проверить единственность решения.
Для соответствующей однородной СЛАУ проверить существование нетривиального решения. В случае, если оно существует,
найти размерность пространства решений и составить ФСР и общее решение однородной  и неоднородной СЛАУ.


\begin{align*}
 A = \left[\begin{matrix}-3 & -8 & -2 & -8 & 4\\-22 & -42 & -38 & -37 & -4\\-122 & -242 & -198 & -217 & -4\\98 & 178 & 182 & 153 & 36\\306 & 566 & 554 & 491 & 92\end{matrix}\right],
\ b = \left[\begin{matrix}11\\-66\\-286\\374\\1078\end{matrix}\right]. 
 \end{align*}

Вариант N 95
Дана СЛАУ $AX = b$,
Проверить совместность по теореме Кронекера-Капелли. Если СЛАУ совместна, проверить единственность решения.
Для соответствующей однородной СЛАУ проверить существование нетривиального решения. В случае, если оно существует,
найти размерность пространства решений и составить ФСР и общее решение однородной  и неоднородной СЛАУ.


\begin{align*}
 A = \left[\begin{matrix}9 & -1 & 2 & -6 & -6\\7 & 1 & 22 & -22 & 2\\55 & 1 & 94 & -106 & -10\\-1 & -7 & -82 & 70 & -26\\-30 & -18 & -252 & 228 & -60\end{matrix}\right],
\ b = \left[\begin{matrix}-97\\-243\\-1263\\681\\2334\end{matrix}\right]. 
 \end{align*}

Вариант N 96
Дана СЛАУ $AX = b$,
Проверить совместность по теореме Кронекера-Капелли. Если СЛАУ совместна, проверить единственность решения.
Для соответствующей однородной СЛАУ проверить существование нетривиального решения. В случае, если оно существует,
найти размерность пространства решений и составить ФСР и общее решение однородной  и неоднородной СЛАУ.


\begin{align*}
 A = \left[\begin{matrix}4 & 6 & 2 & 8 & 3\\-4 & -4 & 8 & -7 & 6\\-8 & -2 & 46 & -11 & 39\\32 & 38 & -34 & 59 & -21\\-2 & -1 & -9 & 6 & -5\end{matrix}\right],
\ b = \left[\begin{matrix}-46\\-113\\-703\\427\\81\end{matrix}\right]. 
 \end{align*}

Вариант N 97
Дана СЛАУ $AX = b$,
Проверить совместность по теореме Кронекера-Капелли. Если СЛАУ совместна, проверить единственность решения.
Для соответствующей однородной СЛАУ проверить существование нетривиального решения. В случае, если оно существует,
найти размерность пространства решений и составить ФСР и общее решение однородной  и неоднородной СЛАУ.


\begin{align*}
 A = \left[\begin{matrix}7 & 5 & -3 & 8 & -8\\6 & -2 & -6 & -7 & -7\\-3 & 23 & 15 & 52 & 4\\9 & -8 & -4 & 2 & 7\\-8 & 0 & 0 & 7 & -9\end{matrix}\right],
\ b = \left[\begin{matrix}69\\-57\\435\\-93\\99\end{matrix}\right]. 
 \end{align*}

Вариант N 98
Дана СЛАУ $AX = b$,
Проверить совместность по теореме Кронекера-Капелли. Если СЛАУ совместна, проверить единственность решения.
Для соответствующей однородной СЛАУ проверить существование нетривиального решения. В случае, если оно существует,
найти размерность пространства решений и составить ФСР и общее решение однородной  и неоднородной СЛАУ.


\begin{align*}
 A = \left[\begin{matrix}1 & 5 & -1 & 5 & -8\\1 & -8 & 3 & 1 & -8\\9 & -20 & 11 & 25 & -72\\-1 & 60 & -19 & 15 & 8\\-6 & 4 & 7 & -4 & 0\end{matrix}\right],
\ b = \left[\begin{matrix}10\\-111\\-515\\595\\-51\end{matrix}\right]. 
 \end{align*}

Вариант N 99
Дана СЛАУ $AX = b$,
Проверить совместность по теореме Кронекера-Капелли. Если СЛАУ совместна, проверить единственность решения.
Для соответствующей однородной СЛАУ проверить существование нетривиального решения. В случае, если оно существует,
найти размерность пространства решений и составить ФСР и общее решение однородной  и неоднородной СЛАУ.


\begin{align*}
 A = \left[\begin{matrix}5 & 9 & 1 & -1 & -7\\1 & 0 & -8 & 1 & 9\\24 & 36 & -28 & 0 & 8\\16 & 36 & 36 & -8 & -64\\-6 & -5 & 3 & -1 & 3\end{matrix}\right],
\ b = \left[\begin{matrix}63\\37\\400\\104\\-62\end{matrix}\right]. 
 \end{align*}

Вариант N 100
Дана СЛАУ $AX = b$,
Проверить совместность по теореме Кронекера-Капелли. Если СЛАУ совместна, проверить единственность решения.
Для соответствующей однородной СЛАУ проверить существование нетривиального решения. В случае, если оно существует,
найти размерность пространства решений и составить ФСР и общее решение однородной  и неоднородной СЛАУ.


\begin{align*}
 A = \left[\begin{matrix}-9 & 3 & -4 & -7 & -8\\0 & 4 & -5 & 3 & -9\\-36 & -8 & 9 & -43 & 13\\-2 & -7 & 3 & 4 & 4\\4 & -5 & 8 & 3 & 5\end{matrix}\right],
\ b = \left[\begin{matrix}-1\\75\\-379\\-64\\-45\end{matrix}\right]. 
 \end{align*}

Вариант N 101
Дана СЛАУ $AX = b$,
Проверить совместность по теореме Кронекера-Капелли. Если СЛАУ совместна, проверить единственность решения.
Для соответствующей однородной СЛАУ проверить существование нетривиального решения. В случае, если оно существует,
найти размерность пространства решений и составить ФСР и общее решение однородной  и неоднородной СЛАУ.


\begin{align*}
 A = \left[\begin{matrix}-3 & 5 & -1 & -4 & -9\\-3 & 0 & -1 & 8 & -3\\6 & 15 & 2 & -52 & -12\\-4 & 8 & -6 & 2 & -9\\-9 & -4 & -1 & -8 & 9\end{matrix}\right],
\ b = \left[\begin{matrix}48\\45\\-81\\13\\-93\end{matrix}\right]. 
 \end{align*}

Вариант N 102
Дана СЛАУ $AX = b$,
Проверить совместность по теореме Кронекера-Капелли. Если СЛАУ совместна, проверить единственность решения.
Для соответствующей однородной СЛАУ проверить существование нетривиального решения. В случае, если оно существует,
найти размерность пространства решений и составить ФСР и общее решение однородной  и неоднородной СЛАУ.


\begin{align*}
 A = \left[\begin{matrix}-2 & -8 & 5 & -4 & 3\\6 & -56 & 23 & -8 & 21\\18 & -248 & 107 & -44 & 93\\-30 & 200 & -77 & 20 & -75\\-84 & 624 & -246 & 72 & -234\end{matrix}\right],
\ b = \left[\begin{matrix}66\\282\\1326\\-930\\-2988\end{matrix}\right]. 
 \end{align*}

Вариант N 103
Дана СЛАУ $AX = b$,
Проверить совместность по теореме Кронекера-Капелли. Если СЛАУ совместна, проверить единственность решения.
Для соответствующей однородной СЛАУ проверить существование нетривиального решения. В случае, если оно существует,
найти размерность пространства решений и составить ФСР и общее решение однородной  и неоднородной СЛАУ.


\begin{align*}
 A = \left[\begin{matrix}4 & 8 & -5 & 5 & 5\\-3 & -1 & -20 & 30 & 35\\-3 & 19 & -115 & 165 & 190\\27 & 29 & 85 & -135 & -160\\69 & 63 & 270 & -420 & -495\end{matrix}\right],
\ b = \left[\begin{matrix}72\\191\\1171\\-739\\-2433\end{matrix}\right]. 
 \end{align*}

Вариант N 104
Дана СЛАУ $AX = b$,
Проверить совместность по теореме Кронекера-Капелли. Если СЛАУ совместна, проверить единственность решения.
Для соответствующей однородной СЛАУ проверить существование нетривиального решения. В случае, если оно существует,
найти размерность пространства решений и составить ФСР и общее решение однородной  и неоднородной СЛАУ.


\begin{align*}
 A = \left[\begin{matrix}-2 & 6 & 4 & 8 & 1\\6 & 3 & -7 & -3 & -4\\22 & 39 & -19 & 17 & -16\\-38 & 9 & 51 & 47 & 24\\8 & 9 & -7 & 4 & -2\end{matrix}\right],
\ b = \left[\begin{matrix}-33\\-14\\-202\\-62\\-78\end{matrix}\right]. 
 \end{align*}

Вариант N 105
Дана СЛАУ $AX = b$,
Проверить совместность по теореме Кронекера-Капелли. Если СЛАУ совместна, проверить единственность решения.
Для соответствующей однородной СЛАУ проверить существование нетривиального решения. В случае, если оно существует,
найти размерность пространства решений и составить ФСР и общее решение однородной  и неоднородной СЛАУ.


\begin{align*}
 A = \left[\begin{matrix}9 & 1 & -3 & 5 & 8\\52 & -28 & 12 & 36 & 4\\244 & -108 & 36 & 164 & 48\\-172 & 116 & -60 & -124 & 16\\-552 & 344 & -168 & -392 & 16\end{matrix}\right],
\ b = \left[\begin{matrix}-109\\-732\\-3364\\2492\\7912\end{matrix}\right]. 
 \end{align*}

Вариант N 106
Дана СЛАУ $AX = b$,
Проверить совместность по теореме Кронекера-Капелли. Если СЛАУ совместна, проверить единственность решения.
Для соответствующей однородной СЛАУ проверить существование нетривиального решения. В случае, если оно существует,
найти размерность пространства решений и составить ФСР и общее решение однородной  и неоднородной СЛАУ.


\begin{align*}
 A = \left[\begin{matrix}-8 & -7 & 8 & 8 & 7\\-2 & -53 & 37 & 17 & 8\\-42 & -293 & 217 & 117 & 68\\-22 & 237 & -153 & -53 & -12\\-34 & 739 & -491 & -191 & -64\end{matrix}\right],
\ b = \left[\begin{matrix}110\\380\\2340\\-1460\\-4820\end{matrix}\right]. 
 \end{align*}

Вариант N 107
Дана СЛАУ $AX = b$,
Проверить совместность по теореме Кронекера-Капелли. Если СЛАУ совместна, проверить единственность решения.
Для соответствующей однородной СЛАУ проверить существование нетривиального решения. В случае, если оно существует,
найти размерность пространства решений и составить ФСР и общее решение однородной  и неоднородной СЛАУ.


\begin{align*}
 A = \left[\begin{matrix}8 & 5 & 8 & 3 & 2\\64 & 35 & 44 & -11 & -19\\344 & 190 & 244 & -46 & -89\\-296 & -160 & -196 & 64 & 101\\-912 & -495 & -612 & 183 & 297\end{matrix}\right],
\ b = \left[\begin{matrix}-116\\-693\\-3813\\3117\\9699\end{matrix}\right]. 
 \end{align*}

Вариант N 108
Дана СЛАУ $AX = b$,
Проверить совместность по теореме Кронекера-Капелли. Если СЛАУ совместна, проверить единственность решения.
Для соответствующей однородной СЛАУ проверить существование нетривиального решения. В случае, если оно существует,
найти размерность пространства решений и составить ФСР и общее решение однородной  и неоднородной СЛАУ.


\begin{align*}
 A = \left[\begin{matrix}-1 & 8 & 1 & 3 & -8\\-1 & -5 & 2 & -9 & 9\\1 & 44 & -5 & 45 & -60\\1 & -8 & -9 & 2 & 0\\-1 & 5 & -6 & -7 & -4\end{matrix}\right],
\ b = \left[\begin{matrix}50\\-78\\462\\20\\-16\end{matrix}\right]. 
 \end{align*}

Вариант N 109
Дана СЛАУ $AX = b$,
Проверить совместность по теореме Кронекера-Капелли. Если СЛАУ совместна, проверить единственность решения.
Для соответствующей однородной СЛАУ проверить существование нетривиального решения. В случае, если оно существует,
найти размерность пространства решений и составить ФСР и общее решение однородной  и неоднородной СЛАУ.


\begin{align*}
 A = \left[\begin{matrix}-1 & 8 & -8 & -5 & 5\\-31 & -8 & -44 & -23 & -5\\-127 & -8 & -200 & -107 & -5\\121 & 56 & 152 & 77 & 35\\366 & 144 & 480 & 246 & 90\end{matrix}\right],
\ b = \left[\begin{matrix}101\\607\\2731\\-2125\\-6678\end{matrix}\right]. 
 \end{align*}

Вариант N 110
Дана СЛАУ $AX = b$,
Проверить совместность по теореме Кронекера-Капелли. Если СЛАУ совместна, проверить единственность решения.
Для соответствующей однородной СЛАУ проверить существование нетривиального решения. В случае, если оно существует,
найти размерность пространства решений и составить ФСР и общее решение однородной  и неоднородной СЛАУ.


\begin{align*}
 A = \left[\begin{matrix}9 & 6 & 3 & 9 & 5\\52 & 48 & 0 & 44 & 48\\244 & 216 & 12 & 212 & 212\\-172 & -168 & 12 & -140 & -172\\-552 & -528 & 24 & -456 & -536\end{matrix}\right],
\ b = \left[\begin{matrix}76\\636\\2848\\-2240\\-7024\end{matrix}\right]. 
 \end{align*}

Вариант N 111
Дана СЛАУ $AX = b$,
Проверить совместность по теореме Кронекера-Капелли. Если СЛАУ совместна, проверить единственность решения.
Для соответствующей однородной СЛАУ проверить существование нетривиального решения. В случае, если оно существует,
найти размерность пространства решений и составить ФСР и общее решение однородной  и неоднородной СЛАУ.


\begin{align*}
 A = \left[\begin{matrix}-6 & 3 & -1 & 9 & 4\\5 & 0 & -8 & 1 & 4\\-4 & 12 & -36 & 40 & 32\\-44 & 12 & 28 & 32 & 0\\6 & 4 & -7 & 1 & -6\end{matrix}\right],
\ b = \left[\begin{matrix}32\\-45\\-52\\308\\-63\end{matrix}\right]. 
 \end{align*}

Вариант N 112
Дана СЛАУ $AX = b$,
Проверить совместность по теореме Кронекера-Капелли. Если СЛАУ совместна, проверить единственность решения.
Для соответствующей однородной СЛАУ проверить существование нетривиального решения. В случае, если оно существует,
найти размерность пространства решений и составить ФСР и общее решение однородной  и неоднородной СЛАУ.


\begin{align*}
 A = \left[\begin{matrix}8 & -4 & -8 & 0 & 0\\-9 & -9 & 3 & -1 & -7\\-4 & -52 & -20 & -4 & -28\\68 & 20 & -44 & 4 & 28\\-3 & -6 & 7 & -8 & -4\end{matrix}\right],
\ b = \left[\begin{matrix}84\\-111\\-108\\780\\-4\end{matrix}\right]. 
 \end{align*}

Вариант N 113
Дана СЛАУ $AX = b$,
Проверить совместность по теореме Кронекера-Капелли. Если СЛАУ совместна, проверить единственность решения.
Для соответствующей однородной СЛАУ проверить существование нетривиального решения. В случае, если оно существует,
найти размерность пространства решений и составить ФСР и общее решение однородной  и неоднородной СЛАУ.


\begin{align*}
 A = \left[\begin{matrix}-5 & 8 & 6 & -4 & -9\\15 & 19 & 53 & -12 & -42\\60 & 119 & 283 & -72 & -237\\-90 & -71 & -247 & 48 & 183\\-255 & -237 & -759 & 156 & 576\end{matrix}\right],
\ b = \left[\begin{matrix}-33\\166\\731\\-929\\-2688\end{matrix}\right]. 
 \end{align*}

Вариант N 114
Дана СЛАУ $AX = b$,
Проверить совместность по теореме Кронекера-Капелли. Если СЛАУ совместна, проверить единственность решения.
Для соответствующей однородной СЛАУ проверить существование нетривиального решения. В случае, если оно существует,
найти размерность пространства решений и составить ФСР и общее решение однородной  и неоднородной СЛАУ.


\begin{align*}
 A = \left[\begin{matrix}1 & -9 & -8 & 0 & 3\\8 & -32 & -39 & -25 & -6\\43 & -187 & -219 & -125 & -21\\-37 & 133 & 171 & 125 & 39\\-114 & 426 & 537 & 375 & 108\end{matrix}\right],
\ b = \left[\begin{matrix}-22\\-226\\-1196\\1064\\3258\end{matrix}\right]. 
 \end{align*}

Вариант N 115
Дана СЛАУ $AX = b$,
Проверить совместность по теореме Кронекера-Капелли. Если СЛАУ совместна, проверить единственность решения.
Для соответствующей однородной СЛАУ проверить существование нетривиального решения. В случае, если оно существует,
найти размерность пространства решений и составить ФСР и общее решение однородной  и неоднородной СЛАУ.


\begin{align*}
 A = \left[\begin{matrix}-2 & -1 & -8 & 6 & 1\\-6 & 2 & -7 & -4 & -4\\-38 & 6 & -67 & 4 & -16\\22 & -14 & 3 & 44 & 24\\5 & 0 & 4 & -2 & -1\end{matrix}\right],
\ b = \left[\begin{matrix}-9\\75\\339\\-411\\14\end{matrix}\right]. 
 \end{align*}

Вариант N 116
Дана СЛАУ $AX = b$,
Проверить совместность по теореме Кронекера-Капелли. Если СЛАУ совместна, проверить единственность решения.
Для соответствующей однородной СЛАУ проверить существование нетривиального решения. В случае, если оно существует,
найти размерность пространства решений и составить ФСР и общее решение однородной  и неоднородной СЛАУ.


\begin{align*}
 A = \left[\begin{matrix}8 & -1 & 7 & 0 & 2\\-9 & 2 & 9 & 4 & -5\\-13 & 6 & 73 & 20 & -17\\77 & -14 & -17 & -20 & 33\\4 & 8 & -4 & 2 & -1\end{matrix}\right],
\ b = \left[\begin{matrix}20\\94\\550\\-390\\-96\end{matrix}\right]. 
 \end{align*}

Вариант N 117
Дана СЛАУ $AX = b$,
Проверить совместность по теореме Кронекера-Капелли. Если СЛАУ совместна, проверить единственность решения.
Для соответствующей однородной СЛАУ проверить существование нетривиального решения. В случае, если оно существует,
найти размерность пространства решений и составить ФСР и общее решение однородной  и неоднородной СЛАУ.


\begin{align*}
 A = \left[\begin{matrix}-7 & 5 & 1 & 0 & 8\\-8 & 4 & -4 & 0 & -4\\-60 & 36 & -12 & 0 & 16\\4 & 4 & 20 & 0 & 48\\40 & -8 & 56 & 0 & 112\end{matrix}\right],
\ b = \left[\begin{matrix}61\\-48\\52\\436\\1064\end{matrix}\right]. 
 \end{align*}

Вариант N 118
Дана СЛАУ $AX = b$,
Проверить совместность по теореме Кронекера-Капелли. Если СЛАУ совместна, проверить единственность решения.
Для соответствующей однородной СЛАУ проверить существование нетривиального решения. В случае, если оно существует,
найти размерность пространства решений и составить ФСР и общее решение однородной  и неоднородной СЛАУ.


\begin{align*}
 A = \left[\begin{matrix}-6 & 8 & 4 & -3 & -5\\-2 & 2 & -9 & -3 & -1\\-16 & 24 & 52 & 0 & -16\\4 & 9 & -5 & -3 & -7\\8 & 9 & 4 & -3 & -2\end{matrix}\right],
\ b = \left[\begin{matrix}26\\58\\-128\\-8\\-33\end{matrix}\right]. 
 \end{align*}

Вариант N 119
Дана СЛАУ $AX = b$,
Проверить совместность по теореме Кронекера-Капелли. Если СЛАУ совместна, проверить единственность решения.
Для соответствующей однородной СЛАУ проверить существование нетривиального решения. В случае, если оно существует,
найти размерность пространства решений и составить ФСР и общее решение однородной  и неоднородной СЛАУ.


\begin{align*}
 A = \left[\begin{matrix}4 & 2 & -8 & -9 & -2\\-7 & -5 & -8 & 3 & -6\\47 & 31 & 16 & -42 & 24\\7 & -3 & 3 & -7 & 1\\8 & -8 & -3 & -3 & 3\end{matrix}\right],
\ b = \left[\begin{matrix}6\\44\\-202\\55\\101\end{matrix}\right]. 
 \end{align*}

Вариант N 120
Дана СЛАУ $AX = b$,
Проверить совместность по теореме Кронекера-Капелли. Если СЛАУ совместна, проверить единственность решения.
Для соответствующей однородной СЛАУ проверить существование нетривиального решения. В случае, если оно существует,
найти размерность пространства решений и составить ФСР и общее решение однородной  и неоднородной СЛАУ.


\begin{align*}
 A = \left[\begin{matrix}8 & 8 & -3 & 8 & -8\\-8 & -2 & 9 & 3 & 4\\56 & 32 & -45 & 12 & -40\\6 & -1 & 0 & 7 & 9\\6 & -6 & 4 & -5 & 0\end{matrix}\right],
\ b = \left[\begin{matrix}-16\\-4\\-32\\23\\104\end{matrix}\right]. 
 \end{align*}

Вариант N 121
Дана СЛАУ $AX = b$,
Проверить совместность по теореме Кронекера-Капелли. Если СЛАУ совместна, проверить единственность решения.
Для соответствующей однородной СЛАУ проверить существование нетривиального решения. В случае, если оно существует,
найти размерность пространства решений и составить ФСР и общее решение однородной  и неоднородной СЛАУ.


\begin{align*}
 A = \left[\begin{matrix}9 & 5 & -3 & -3 & -5\\8 & -5 & -8 & -4 & 7\\-4 & 45 & 28 & 8 & -55\\0 & -7 & 8 & -3 & -4\\4 & -7 & 4 & -7 & 4\end{matrix}\right],
\ b = \left[\begin{matrix}5\\191\\-935\\18\\138\end{matrix}\right]. 
 \end{align*}

Вариант N 122
Дана СЛАУ $AX = b$,
Проверить совместность по теореме Кронекера-Капелли. Если СЛАУ совместна, проверить единственность решения.
Для соответствующей однородной СЛАУ проверить существование нетривиального решения. В случае, если оно существует,
найти размерность пространства решений и составить ФСР и общее решение однородной  и неоднородной СЛАУ.


\begin{align*}
 A = \left[\begin{matrix}-6 & 4 & 1 & -3 & 1\\-7 & 2 & -8 & -5 & -2\\-46 & 20 & -29 & -29 & -5\\10 & 4 & 35 & 11 & 11\\-6 & 2 & 9 & 5 & -6\end{matrix}\right],
\ b = \left[\begin{matrix}-49\\-33\\-279\\-15\\-96\end{matrix}\right]. 
 \end{align*}

Вариант N 123
Дана СЛАУ $AX = b$,
Проверить совместность по теореме Кронекера-Капелли. Если СЛАУ совместна, проверить единственность решения.
Для соответствующей однородной СЛАУ проверить существование нетривиального решения. В случае, если оно существует,
найти размерность пространства решений и составить ФСР и общее решение однородной  и неоднородной СЛАУ.


\begin{align*}
 A = \left[\begin{matrix}5 & -1 & 0 & -5 & 6\\-3 & 2 & 6 & 8 & 1\\35 & -14 & -30 & -60 & 19\\4 & -2 & 6 & -9 & -5\\-8 & 4 & 6 & 3 & 5\end{matrix}\right],
\ b = \left[\begin{matrix}61\\-72\\604\\73\\-59\end{matrix}\right]. 
 \end{align*}

Вариант N 124
Дана СЛАУ $AX = b$,
Проверить совместность по теореме Кронекера-Капелли. Если СЛАУ совместна, проверить единственность решения.
Для соответствующей однородной СЛАУ проверить существование нетривиального решения. В случае, если оно существует,
найти размерность пространства решений и составить ФСР и общее решение однородной  и неоднородной СЛАУ.


\begin{align*}
 A = \left[\begin{matrix}0 & 7 & -9 & -9 & -9\\-35 & -9 & -57 & -37 & 3\\-175 & -24 & -312 & -212 & -12\\175 & 66 & 258 & 158 & -42\\525 & 177 & 801 & 501 & -99\end{matrix}\right],
\ b = \left[\begin{matrix}36\\448\\2348\\-2132\\-6504\end{matrix}\right]. 
 \end{align*}

Вариант N 125
Дана СЛАУ $AX = b$,
Проверить совместность по теореме Кронекера-Капелли. Если СЛАУ совместна, проверить единственность решения.
Для соответствующей однородной СЛАУ проверить существование нетривиального решения. В случае, если оно существует,
найти размерность пространства решений и составить ФСР и общее решение однородной  и неоднородной СЛАУ.


\begin{align*}
 A = \left[\begin{matrix}5 & 3 & 4 & -6 & 6\\8 & -1 & 9 & -8 & 0\\-25 & 14 & -33 & 22 & 18\\7 & -3 & -1 & 6 & -8\\3 & -1 & 3 & 2 & -9\end{matrix}\right],
\ b = \left[\begin{matrix}-119\\-67\\-22\\74\\75\end{matrix}\right]. 
 \end{align*}

Вариант N 126
Дана СЛАУ $AX = b$,
Проверить совместность по теореме Кронекера-Капелли. Если СЛАУ совместна, проверить единственность решения.
Для соответствующей однородной СЛАУ проверить существование нетривиального решения. В случае, если оно существует,
найти размерность пространства решений и составить ФСР и общее решение однородной  и неоднородной СЛАУ.


\begin{align*}
 A = \left[\begin{matrix}-4 & -8 & -9 & 0 & -1\\-37 & 16 & -72 & 5 & 27\\-197 & 56 & -387 & 25 & 132\\173 & -104 & 333 & -25 & -138\\531 & -288 & 1026 & -75 & -411\end{matrix}\right],
\ b = \left[\begin{matrix}-76\\-193\\-1193\\737\\2439\end{matrix}\right]. 
 \end{align*}

Вариант N 127
Дана СЛАУ $AX = b$,
Проверить совместность по теореме Кронекера-Капелли. Если СЛАУ совместна, проверить единственность решения.
Для соответствующей однородной СЛАУ проверить существование нетривиального решения. В случае, если оно существует,
найти размерность пространства решений и составить ФСР и общее решение однородной  и неоднородной СЛАУ.


\begin{align*}
 A = \left[\begin{matrix}0 & 8 & -7 & -9 & -2\\9 & -7 & -3 & -8 & 7\\45 & -3 & -43 & -76 & 27\\-45 & 67 & -13 & 4 & -43\\7 & 3 & 4 & -3 & 8\end{matrix}\right],
\ b = \left[\begin{matrix}63\\-110\\-298\\802\\-101\end{matrix}\right]. 
 \end{align*}

Вариант N 128
Дана СЛАУ $AX = b$,
Проверить совместность по теореме Кронекера-Капелли. Если СЛАУ совместна, проверить единственность решения.
Для соответствующей однородной СЛАУ проверить существование нетривиального решения. В случае, если оно существует,
найти размерность пространства решений и составить ФСР и общее решение однородной  и неоднородной СЛАУ.


\begin{align*}
 A = \left[\begin{matrix}-2 & 1 & 0 & -8 & 0\\-8 & -5 & 1 & -4 & 1\\-46 & -22 & 5 & -44 & 5\\34 & 28 & -5 & -4 & -5\\7 & 1 & 0 & -5 & 7\end{matrix}\right],
\ b = \left[\begin{matrix}-75\\-148\\-965\\515\\16\end{matrix}\right]. 
 \end{align*}

Вариант N 129
Дана СЛАУ $AX = b$,
Проверить совместность по теореме Кронекера-Капелли. Если СЛАУ совместна, проверить единственность решения.
Для соответствующей однородной СЛАУ проверить существование нетривиального решения. В случае, если оно существует,
найти размерность пространства решений и составить ФСР и общее решение однородной  и неоднородной СЛАУ.


\begin{align*}
 A = \left[\begin{matrix}6 & 5 & 8 & 7 & -4\\-8 & 4 & -4 & -2 & -1\\-22 & 35 & 4 & 11 & -17\\58 & -5 & 44 & 31 & -7\\1 & 0 & -5 & 6 & -5\end{matrix}\right],
\ b = \left[\begin{matrix}-5\\-96\\-495\\465\\-86\end{matrix}\right]. 
 \end{align*}

Вариант N 130
Дана СЛАУ $AX = b$,
Проверить совместность по теореме Кронекера-Капелли. Если СЛАУ совместна, проверить единственность решения.
Для соответствующей однородной СЛАУ проверить существование нетривиального решения. В случае, если оно существует,
найти размерность пространства решений и составить ФСР и общее решение однородной  и неоднородной СЛАУ.


\begin{align*}
 A = \left[\begin{matrix}9 & 1 & 7 & 9 & 7\\36 & -26 & 68 & 31 & 18\\216 & -126 & 368 & 191 & 118\\-144 & 134 & -312 & -119 & -62\\-468 & 398 & -964 & -393 & -214\end{matrix}\right],
\ b = \left[\begin{matrix}-127\\-638\\-3698\\2682\\8554\end{matrix}\right]. 
 \end{align*}

Вариант N 131
Дана СЛАУ $AX = b$,
Проверить совместность по теореме Кронекера-Капелли. Если СЛАУ совместна, проверить единственность решения.
Для соответствующей однородной СЛАУ проверить существование нетривиального решения. В случае, если оно существует,
найти размерность пространства решений и составить ФСР и общее решение однородной  и неоднородной СЛАУ.


\begin{align*}
 A = \left[\begin{matrix}0 & -7 & -6 & 2 & -2\\-24 & -5 & -22 & -10 & 14\\-96 & -41 & -106 & -34 & 50\\96 & -1 & 70 & 46 & -62\\288 & 18 & 228 & 132 & -180\end{matrix}\right],
\ b = \left[\begin{matrix}14\\62\\290\\-206\\-660\end{matrix}\right]. 
 \end{align*}

Вариант N 132
Дана СЛАУ $AX = b$,
Проверить совместность по теореме Кронекера-Капелли. Если СЛАУ совместна, проверить единственность решения.
Для соответствующей однородной СЛАУ проверить существование нетривиального решения. В случае, если оно существует,
найти размерность пространства решений и составить ФСР и общее решение однородной  и неоднородной СЛАУ.


\begin{align*}
 A = \left[\begin{matrix}-3 & -9 & 2 & 8 & 7\\-45 & -23 & 14 & 4 & 9\\-189 & -119 & 62 & 40 & 57\\171 & 65 & -50 & 8 & -15\\522 & 222 & -156 & 0 & -66\end{matrix}\right],
\ b = \left[\begin{matrix}57\\-181\\-553\\895\\2514\end{matrix}\right]. 
 \end{align*}

Вариант N 133
Дана СЛАУ $AX = b$,
Проверить совместность по теореме Кронекера-Капелли. Если СЛАУ совместна, проверить единственность решения.
Для соответствующей однородной СЛАУ проверить существование нетривиального решения. В случае, если оно существует,
найти размерность пространства решений и составить ФСР и общее решение однородной  и неоднородной СЛАУ.


\begin{align*}
 A = \left[\begin{matrix}8 & -6 & 0 & -5 & -9\\-6 & -4 & -5 & 0 & -2\\8 & -40 & -20 & -20 & -44\\56 & -8 & 20 & -20 & -28\\3 & 9 & -7 & 1 & -5\end{matrix}\right],
\ b = \left[\begin{matrix}-105\\-31\\-544\\-296\\4\end{matrix}\right]. 
 \end{align*}

Вариант N 134
Дана СЛАУ $AX = b$,
Проверить совместность по теореме Кронекера-Капелли. Если СЛАУ совместна, проверить единственность решения.
Для соответствующей однородной СЛАУ проверить существование нетривиального решения. В случае, если оно существует,
найти размерность пространства решений и составить ФСР и общее решение однородной  и неоднородной СЛАУ.


\begin{align*}
 A = \left[\begin{matrix}-6 & 5 & 6 & -5 & 4\\-3 & -2 & -6 & 8 & 8\\-6 & 23 & 42 & -47 & -20\\5 & 5 & 5 & 4 & 4\\-4 & -6 & -8 & -4 & -8\end{matrix}\right],
\ b = \left[\begin{matrix}-54\\141\\-726\\29\\-52\end{matrix}\right]. 
 \end{align*}

Вариант N 135
Дана СЛАУ $AX = b$,
Проверить совместность по теореме Кронекера-Капелли. Если СЛАУ совместна, проверить единственность решения.
Для соответствующей однородной СЛАУ проверить существование нетривиального решения. В случае, если оно существует,
найти размерность пространства решений и составить ФСР и общее решение однородной  и неоднородной СЛАУ.


\begin{align*}
 A = \left[\begin{matrix}-9 & -6 & 9 & 0 & 4\\-37 & -8 & 42 & -15 & -28\\-212 & -58 & 237 & -75 & -128\\158 & 22 & -183 & 75 & 152\\501 & 84 & -576 & 225 & 444\end{matrix}\right],
\ b = \left[\begin{matrix}51\\-187\\-782\\1088\\3111\end{matrix}\right]. 
 \end{align*}

Вариант N 136
Дана СЛАУ $AX = b$,
Проверить совместность по теореме Кронекера-Капелли. Если СЛАУ совместна, проверить единственность решения.
Для соответствующей однородной СЛАУ проверить существование нетривиального решения. В случае, если оно существует,
найти размерность пространства решений и составить ФСР и общее решение однородной  и неоднородной СЛАУ.


\begin{align*}
 A = \left[\begin{matrix}-5 & 4 & -5 & -4 & -7\\-9 & 4 & -7 & 8 & 1\\-60 & 32 & -50 & 28 & -16\\30 & -8 & 20 & -52 & -26\\-2 & -2 & -9 & 0 & -1\end{matrix}\right],
\ b = \left[\begin{matrix}-85\\37\\-70\\-440\\50\end{matrix}\right]. 
 \end{align*}

Вариант N 137
Дана СЛАУ $AX = b$,
Проверить совместность по теореме Кронекера-Капелли. Если СЛАУ совместна, проверить единственность решения.
Для соответствующей однородной СЛАУ проверить существование нетривиального решения. В случае, если оно существует,
найти размерность пространства решений и составить ФСР и общее решение однородной  и неоднородной СЛАУ.


\begin{align*}
 A = \left[\begin{matrix}5 & -5 & -2 & 4 & -5\\9 & -2 & -8 & -3 & 4\\-25 & -10 & 32 & 31 & -40\\-9 & 3 & 6 & -7 & 8\\-1 & -5 & 7 & 8 & 8\end{matrix}\right],
\ b = \left[\begin{matrix}-14\\-23\\59\\-14\\-6\end{matrix}\right]. 
 \end{align*}

Вариант N 138
Дана СЛАУ $AX = b$,
Проверить совместность по теореме Кронекера-Капелли. Если СЛАУ совместна, проверить единственность решения.
Для соответствующей однородной СЛАУ проверить существование нетривиального решения. В случае, если оно существует,
найти размерность пространства решений и составить ФСР и общее решение однородной  и неоднородной СЛАУ.


\begin{align*}
 A = \left[\begin{matrix}2 & 5 & -1 & 6 & 2\\-3 & -2 & -6 & 3 & 4\\-9 & 5 & -33 & 33 & 26\\21 & 25 & 27 & 3 & -14\\-3 & -7 & 6 & -9 & 9\end{matrix}\right],
\ b = \left[\begin{matrix}-61\\-87\\-618\\252\\52\end{matrix}\right]. 
 \end{align*}

Вариант N 139
Дана СЛАУ $AX = b$,
Проверить совместность по теореме Кронекера-Капелли. Если СЛАУ совместна, проверить единственность решения.
Для соответствующей однородной СЛАУ проверить существование нетривиального решения. В случае, если оно существует,
найти размерность пространства решений и составить ФСР и общее решение однородной  и неоднородной СЛАУ.


\begin{align*}
 A = \left[\begin{matrix}-7 & -8 & 3 & 3 & -2\\3 & -2 & -2 & -3 & 0\\-6 & -34 & -1 & -6 & -6\\-36 & -14 & 19 & 24 & -6\\-2 & -3 & -2 & -1 & 0\end{matrix}\right],
\ b = \left[\begin{matrix}109\\5\\352\\302\\40\end{matrix}\right]. 
 \end{align*}

Вариант N 140
Дана СЛАУ $AX = b$,
Проверить совместность по теореме Кронекера-Капелли. Если СЛАУ совместна, проверить единственность решения.
Для соответствующей однородной СЛАУ проверить существование нетривиального решения. В случае, если оно существует,
найти размерность пространства решений и составить ФСР и общее решение однородной  и неоднородной СЛАУ.


\begin{align*}
 A = \left[\begin{matrix}7 & 1 & -6 & 6 & 6\\3 & -5 & 0 & 6 & -4\\16 & 24 & -24 & 0 & 40\\-4 & -8 & -4 & -8 & -6\\-8 & 7 & 5 & -2 & -2\end{matrix}\right],
\ b = \left[\begin{matrix}-33\\-21\\-48\\72\\18\end{matrix}\right]. 
 \end{align*}

Вариант N 141
Дана СЛАУ $AX = b$,
Проверить совместность по теореме Кронекера-Капелли. Если СЛАУ совместна, проверить единственность решения.
Для соответствующей однородной СЛАУ проверить существование нетривиального решения. В случае, если оно существует,
найти размерность пространства решений и составить ФСР и общее решение однородной  и неоднородной СЛАУ.


\begin{align*}
 A = \left[\begin{matrix}-4 & -3 & -6 & -1 & 9\\4 & -29 & 18 & -27 & 55\\4 & -125 & 54 & -111 & 247\\-28 & 107 & -90 & 105 & -193\\-72 & 330 & -252 & 318 & -606\end{matrix}\right],
\ b = \left[\begin{matrix}24\\-516\\-1992\\2136\\6336\end{matrix}\right]. 
 \end{align*}

Вариант N 142
Дана СЛАУ $AX = b$,
Проверить совместность по теореме Кронекера-Капелли. Если СЛАУ совместна, проверить единственность решения.
Для соответствующей однородной СЛАУ проверить существование нетривиального решения. В случае, если оно существует,
найти размерность пространства решений и составить ФСР и общее решение однородной  и неоднородной СЛАУ.


\begin{align*}
 A = \left[\begin{matrix}4 & -6 & 4 & -3 & 3\\5 & 0 & 4 & -3 & -8\\41 & -24 & 36 & -27 & -28\\-9 & -24 & -4 & 3 & 52\\-5 & 1 & -8 & -8 & -7\end{matrix}\right],
\ b = \left[\begin{matrix}-20\\-110\\-630\\470\\-71\end{matrix}\right]. 
 \end{align*}

Вариант N 143
Дана СЛАУ $AX = b$,
Проверить совместность по теореме Кронекера-Капелли. Если СЛАУ совместна, проверить единственность решения.
Для соответствующей однородной СЛАУ проверить существование нетривиального решения. В случае, если оно существует,
найти размерность пространства решений и составить ФСР и общее решение однородной  и неоднородной СЛАУ.


\begin{align*}
 A = \left[\begin{matrix}1 & -8 & 0 & 1 & 3\\-5 & -8 & -7 & 1 & 5\\23 & 8 & 28 & -1 & -11\\-4 & -4 & 9 & 7 & 4\\6 & 2 & 4 & -9 & -7\end{matrix}\right],
\ b = \left[\begin{matrix}11\\26\\-71\\24\\28\end{matrix}\right]. 
 \end{align*}

Вариант N 144
Дана СЛАУ $AX = b$,
Проверить совместность по теореме Кронекера-Капелли. Если СЛАУ совместна, проверить единственность решения.
Для соответствующей однородной СЛАУ проверить существование нетривиального решения. В случае, если оно существует,
найти размерность пространства решений и составить ФСР и общее решение однородной  и неоднородной СЛАУ.


\begin{align*}
 A = \left[\begin{matrix}-7 & -7 & 8 & 2 & 2\\4 & -6 & 7 & -3 & -1\\-37 & 3 & -4 & 18 & 10\\-8 & 3 & -7 & 8 & -7\\8 & 1 & 6 & -8 & 8\end{matrix}\right],
\ b = \left[\begin{matrix}-99\\-56\\-73\\120\\-118\end{matrix}\right]. 
 \end{align*}

Вариант N 145
Дана СЛАУ $AX = b$,
Проверить совместность по теореме Кронекера-Капелли. Если СЛАУ совместна, проверить единственность решения.
Для соответствующей однородной СЛАУ проверить существование нетривиального решения. В случае, если оно существует,
найти размерность пространства решений и составить ФСР и общее решение однородной  и неоднородной СЛАУ.


\begin{align*}
 A = \left[\begin{matrix}-1 & 3 & 8 & 5 & -3\\-1 & -9 & -2 & -7 & 4\\2 & 54 & 34 & 50 & -29\\-7 & -2 & 1 & -5 & 6\\1 & -3 & -3 & 8 & -9\end{matrix}\right],
\ b = \left[\begin{matrix}38\\-44\\334\\-33\\-48\end{matrix}\right]. 
 \end{align*}

Вариант N 146
Дана СЛАУ $AX = b$,
Проверить совместность по теореме Кронекера-Капелли. Если СЛАУ совместна, проверить единственность решения.
Для соответствующей однородной СЛАУ проверить существование нетривиального решения. В случае, если оно существует,
найти размерность пространства решений и составить ФСР и общее решение однородной  и неоднородной СЛАУ.


\begin{align*}
 A = \left[\begin{matrix}-1 & 4 & -3 & 2 & -7\\-4 & -4 & -1 & 5 & -5\\-20 & 0 & -16 & 28 & -48\\12 & 32 & -8 & -12 & -8\\7 & -1 & 6 & -1 & 8\end{matrix}\right],
\ b = \left[\begin{matrix}56\\20\\304\\144\\5\end{matrix}\right]. 
 \end{align*}

Вариант N 147
Дана СЛАУ $AX = b$,
Проверить совместность по теореме Кронекера-Капелли. Если СЛАУ совместна, проверить единственность решения.
Для соответствующей однородной СЛАУ проверить существование нетривиального решения. В случае, если оно существует,
найти размерность пространства решений и составить ФСР и общее решение однородной  и неоднородной СЛАУ.


\begin{align*}
 A = \left[\begin{matrix}8 & 6 & 9 & -1 & -9\\9 & 1 & -7 & -7 & 3\\68 & 28 & 8 & -32 & -24\\-4 & 20 & 64 & 24 & -48\\7 & 1 & -6 & 9 & 6\end{matrix}\right],
\ b = \left[\begin{matrix}33\\-128\\-380\\644\\2\end{matrix}\right]. 
 \end{align*}

Вариант N 148
Дана СЛАУ $AX = b$,
Проверить совместность по теореме Кронекера-Капелли. Если СЛАУ совместна, проверить единственность решения.
Для соответствующей однородной СЛАУ проверить существование нетривиального решения. В случае, если оно существует,
найти размерность пространства решений и составить ФСР и общее решение однородной  и неоднородной СЛАУ.


\begin{align*}
 A = \left[\begin{matrix}-6 & 6 & 3 & -1 & -2\\6 & -8 & -7 & -9 & -1\\-48 & 56 & 40 & 32 & -4\\3 & 5 & 5 & -3 & 5\\0 & -6 & 2 & 0 & 3\end{matrix}\right],
\ b = \left[\begin{matrix}-6\\9\\-60\\-42\\57\end{matrix}\right]. 
 \end{align*}

Вариант N 149
Дана СЛАУ $AX = b$,
Проверить совместность по теореме Кронекера-Капелли. Если СЛАУ совместна, проверить единственность решения.
Для соответствующей однородной СЛАУ проверить существование нетривиального решения. В случае, если оно существует,
найти размерность пространства решений и составить ФСР и общее решение однородной  и неоднородной СЛАУ.


\begin{align*}
 A = \left[\begin{matrix}-7 & -2 & 8 & -1 & -1\\9 & -5 & -9 & -9 & -3\\8 & -28 & -4 & -40 & -16\\-64 & 12 & 68 & 32 & 8\\-9 & 7 & 7 & 3 & 1\end{matrix}\right],
\ b = \left[\begin{matrix}59\\-115\\-224\\696\\81\end{matrix}\right]. 
 \end{align*}

Вариант N 150
Дана СЛАУ $AX = b$,
Проверить совместность по теореме Кронекера-Капелли. Если СЛАУ совместна, проверить единственность решения.
Для соответствующей однородной СЛАУ проверить существование нетривиального решения. В случае, если оно существует,
найти размерность пространства решений и составить ФСР и общее решение однородной  и неоднородной СЛАУ.


\begin{align*}
 A = \left[\begin{matrix}-4 & -4 & -4 & -6 & 6\\5 & 9 & -4 & 2 & -8\\-32 & -48 & 4 & -26 & 50\\8 & -5 & 0 & 4 & -3\\5 & 5 & -2 & 5 & 2\end{matrix}\right],
\ b = \left[\begin{matrix}32\\-52\\304\\-107\\-2\end{matrix}\right]. 
 \end{align*}

Вариант N 151
Дана СЛАУ $AX = b$,
Проверить совместность по теореме Кронекера-Капелли. Если СЛАУ совместна, проверить единственность решения.
Для соответствующей однородной СЛАУ проверить существование нетривиального решения. В случае, если оно существует,
найти размерность пространства решений и составить ФСР и общее решение однородной  и неоднородной СЛАУ.


\begin{align*}
 A = \left[\begin{matrix}9 & -5 & -9 & -9 & 5\\28 & 0 & 0 & 0 & 0\\148 & -20 & -36 & -36 & 20\\-76 & -20 & -36 & -36 & 20\\-264 & -40 & -72 & -72 & 40\end{matrix}\right],
\ b = \left[\begin{matrix}70\\84\\616\\-56\\-448\end{matrix}\right]. 
 \end{align*}

Вариант N 152
Дана СЛАУ $AX = b$,
Проверить совместность по теореме Кронекера-Капелли. Если СЛАУ совместна, проверить единственность решения.
Для соответствующей однородной СЛАУ проверить существование нетривиального решения. В случае, если оно существует,
найти размерность пространства решений и составить ФСР и общее решение однородной  и неоднородной СЛАУ.


\begin{align*}
 A = \left[\begin{matrix}5 & 1 & -8 & -3 & -5\\27 & -29 & -36 & -9 & 21\\123 & -113 & -168 & -45 & 69\\-93 & 119 & 120 & 27 & -99\\-294 & 354 & 384 & 90 & -282\end{matrix}\right],
\ b = \left[\begin{matrix}-11\\-81\\-357\\291\\906\end{matrix}\right]. 
 \end{align*}

Вариант N 153
Дана СЛАУ $AX = b$,
Проверить совместность по теореме Кронекера-Капелли. Если СЛАУ совместна, проверить единственность решения.
Для соответствующей однородной СЛАУ проверить существование нетривиального решения. В случае, если оно существует,
найти размерность пространства решений и составить ФСР и общее решение однородной  и неоднородной СЛАУ.


\begin{align*}
 A = \left[\begin{matrix}1 & -2 & 7 & 6 & -2\\29 & -13 & -7 & 59 & -33\\149 & -73 & -7 & 319 & -173\\-141 & 57 & 63 & -271 & 157\\-427 & 179 & 161 & -837 & 479\end{matrix}\right],
\ b = \left[\begin{matrix}27\\718\\3698\\-3482\\-10554\end{matrix}\right]. 
 \end{align*}

Вариант N 154
Дана СЛАУ $AX = b$,
Проверить совместность по теореме Кронекера-Капелли. Если СЛАУ совместна, проверить единственность решения.
Для соответствующей однородной СЛАУ проверить существование нетривиального решения. В случае, если оно существует,
найти размерность пространства решений и составить ФСР и общее решение однородной  и неоднородной СЛАУ.


\begin{align*}
 A = \left[\begin{matrix}6 & -4 & -1 & -1 & -2\\-6 & -9 & 2 & 7 & 2\\0 & -52 & 4 & 24 & 0\\48 & 20 & -12 & -32 & -16\\0 & -8 & 8 & 6 & 4\end{matrix}\right],
\ b = \left[\begin{matrix}-69\\10\\-236\\-316\\-72\end{matrix}\right]. 
 \end{align*}

Вариант N 155
Дана СЛАУ $AX = b$,
Проверить совместность по теореме Кронекера-Капелли. Если СЛАУ совместна, проверить единственность решения.
Для соответствующей однородной СЛАУ проверить существование нетривиального решения. В случае, если оно существует,
найти размерность пространства решений и составить ФСР и общее решение однородной  и неоднородной СЛАУ.


\begin{align*}
 A = \left[\begin{matrix}8 & -7 & 9 & 4 & -4\\7 & -23 & 26 & -29 & -1\\67 & -143 & 166 & -129 & -21\\-3 & 87 & -94 & 161 & -11\\-41 & 289 & -318 & 467 & -17\end{matrix}\right],
\ b = \left[\begin{matrix}157\\-52\\368\\888\\2036\end{matrix}\right]. 
 \end{align*}

Вариант N 156
Дана СЛАУ $AX = b$,
Проверить совместность по теореме Кронекера-Капелли. Если СЛАУ совместна, проверить единственность решения.
Для соответствующей однородной СЛАУ проверить существование нетривиального решения. В случае, если оно существует,
найти размерность пространства решений и составить ФСР и общее решение однородной  и неоднородной СЛАУ.


\begin{align*}
 A = \left[\begin{matrix}-8 & -8 & 2 & 7 & 2\\-16 & -52 & -22 & 5 & -26\\-88 & -232 & -82 & 41 & -98\\40 & 184 & 94 & 1 & 110\\144 & 576 & 276 & -18 & 324\end{matrix}\right],
\ b = \left[\begin{matrix}-10\\350\\1370\\-1430\\-4260\end{matrix}\right]. 
 \end{align*}

Вариант N 157
Дана СЛАУ $AX = b$,
Проверить совместность по теореме Кронекера-Капелли. Если СЛАУ совместна, проверить единственность решения.
Для соответствующей однородной СЛАУ проверить существование нетривиального решения. В случае, если оно существует,
найти размерность пространства решений и составить ФСР и общее решение однородной  и неоднородной СЛАУ.


\begin{align*}
 A = \left[\begin{matrix}-6 & 0 & -5 & 4 & 4\\6 & 6 & 9 & 3 & -9\\-48 & -30 & -60 & -3 & 57\\-4 & -9 & 4 & -8 & 4\\3 & 9 & -9 & 3 & 3\end{matrix}\right],
\ b = \left[\begin{matrix}115\\-132\\1005\\-21\\84\end{matrix}\right]. 
 \end{align*}

Вариант N 158
Дана СЛАУ $AX = b$,
Проверить совместность по теореме Кронекера-Капелли. Если СЛАУ совместна, проверить единственность решения.
Для соответствующей однородной СЛАУ проверить существование нетривиального решения. В случае, если оно существует,
найти размерность пространства решений и составить ФСР и общее решение однородной  и неоднородной СЛАУ.


\begin{align*}
 A = \left[\begin{matrix}-2 & -9 & -9 & -9 & -2\\6 & 1 & 0 & 8 & -2\\-38 & -41 & -36 & -76 & 2\\-8 & 0 & 9 & -2 & 3\\2 & -4 & 0 & -6 & 4\end{matrix}\right],
\ b = \left[\begin{matrix}76\\-68\\644\\21\\34\end{matrix}\right]. 
 \end{align*}

Вариант N 159
Дана СЛАУ $AX = b$,
Проверить совместность по теореме Кронекера-Капелли. Если СЛАУ совместна, проверить единственность решения.
Для соответствующей однородной СЛАУ проверить существование нетривиального решения. В случае, если оно существует,
найти размерность пространства решений и составить ФСР и общее решение однородной  и неоднородной СЛАУ.


\begin{align*}
 A = \left[\begin{matrix}-5 & 8 & 0 & 9 & 1\\-5 & 8 & -1 & -8 & 0\\5 & -8 & 4 & 59 & 3\\4 & 4 & 0 & 9 & -3\\-5 & -1 & -6 & 5 & 4\end{matrix}\right],
\ b = \left[\begin{matrix}-78\\-62\\14\\-48\\37\end{matrix}\right]. 
 \end{align*}

Вариант N 160
Дана СЛАУ $AX = b$,
Проверить совместность по теореме Кронекера-Капелли. Если СЛАУ совместна, проверить единственность решения.
Для соответствующей однородной СЛАУ проверить существование нетривиального решения. В случае, если оно существует,
найти размерность пространства решений и составить ФСР и общее решение однородной  и неоднородной СЛАУ.


\begin{align*}
 A = \left[\begin{matrix}6 & -8 & 1 & 5 & 0\\0 & 9 & 2 & 7 & 4\\24 & -68 & -4 & -8 & -16\\8 & -9 & 4 & -1 & -9\\-1 & -6 & -2 & -2 & -5\end{matrix}\right],
\ b = \left[\begin{matrix}49\\58\\-36\\-59\\-28\end{matrix}\right]. 
 \end{align*}

Вариант N 161
Дана СЛАУ $AX = b$,
Проверить совместность по теореме Кронекера-Капелли. Если СЛАУ совместна, проверить единственность решения.
Для соответствующей однородной СЛАУ проверить существование нетривиального решения. В случае, если оно существует,
найти размерность пространства решений и составить ФСР и общее решение однородной  и неоднородной СЛАУ.


\begin{align*}
 A = \left[\begin{matrix}-5 & 7 & 1 & 8 & -9\\3 & -8 & -8 & -8 & -3\\0 & -19 & -37 & -16 & -42\\-30 & 61 & 43 & 64 & -12\\6 & -8 & -8 & 9 & 7\end{matrix}\right],
\ b = \left[\begin{matrix}42\\-30\\-24\\276\\-2\end{matrix}\right]. 
 \end{align*}

Вариант N 162
Дана СЛАУ $AX = b$,
Проверить совместность по теореме Кронекера-Капелли. Если СЛАУ совместна, проверить единственность решения.
Для соответствующей однородной СЛАУ проверить существование нетривиального решения. В случае, если оно существует,
найти размерность пространства решений и составить ФСР и общее решение однородной  и неоднородной СЛАУ.


\begin{align*}
 A = \left[\begin{matrix}-7 & -7 & 3 & 5 & -2\\-40 & -20 & -4 & 8 & -16\\-188 & -108 & -4 & 52 & -72\\132 & 52 & 28 & -12 & 56\\424 & 184 & 72 & -56 & 176\end{matrix}\right],
\ b = \left[\begin{matrix}-33\\-224\\-1028\\764\\2424\end{matrix}\right]. 
 \end{align*}

Вариант N 163
Дана СЛАУ $AX = b$,
Проверить совместность по теореме Кронекера-Капелли. Если СЛАУ совместна, проверить единственность решения.
Для соответствующей однородной СЛАУ проверить существование нетривиального решения. В случае, если оно существует,
найти размерность пространства решений и составить ФСР и общее решение однородной  и неоднородной СЛАУ.


\begin{align*}
 A = \left[\begin{matrix}-6 & -2 & 9 & 1 & -3\\-22 & 10 & 39 & -33 & 27\\-106 & 34 & 183 & -129 & 99\\70 & -46 & -129 & 135 & -117\\228 & -132 & -414 & 402 & -342\end{matrix}\right],
\ b = \left[\begin{matrix}-48\\-524\\-2240\\1952\\6000\end{matrix}\right]. 
 \end{align*}

Вариант N 164
Дана СЛАУ $AX = b$,
Проверить совместность по теореме Кронекера-Капелли. Если СЛАУ совместна, проверить единственность решения.
Для соответствующей однородной СЛАУ проверить существование нетривиального решения. В случае, если оно существует,
найти размерность пространства решений и составить ФСР и общее решение однородной  и неоднородной СЛАУ.


\begin{align*}
 A = \left[\begin{matrix}-2 & 8 & -6 & -7 & -6\\2 & 4 & -5 & 8 & 6\\-14 & 8 & 2 & -53 & -42\\7 & -4 & -5 & 9 & 8\\-1 & 0 & -7 & 1 & 7\end{matrix}\right],
\ b = \left[\begin{matrix}26\\-146\\662\\-184\\-77\end{matrix}\right]. 
 \end{align*}

Вариант N 165
Дана СЛАУ $AX = b$,
Проверить совместность по теореме Кронекера-Капелли. Если СЛАУ совместна, проверить единственность решения.
Для соответствующей однородной СЛАУ проверить существование нетривиального решения. В случае, если оно существует,
найти размерность пространства решений и составить ФСР и общее решение однородной  и неоднородной СЛАУ.


\begin{align*}
 A = \left[\begin{matrix}-5 & 1 & 8 & 0 & -4\\1 & -3 & -8 & -4 & -3\\-16 & -8 & 0 & -16 & -28\\-24 & 16 & 64 & 16 & -4\\-2 & 5 & -5 & 9 & 4\end{matrix}\right],
\ b = \left[\begin{matrix}5\\9\\56\\-16\\85\end{matrix}\right]. 
 \end{align*}

Вариант N 166
Дана СЛАУ $AX = b$,
Проверить совместность по теореме Кронекера-Капелли. Если СЛАУ совместна, проверить единственность решения.
Для соответствующей однородной СЛАУ проверить существование нетривиального решения. В случае, если оно существует,
найти размерность пространства решений и составить ФСР и общее решение однородной  и неоднородной СЛАУ.


\begin{align*}
 A = \left[\begin{matrix}1 & -3 & -2 & 6 & 3\\-9 & -6 & -4 & 5 & 9\\-42 & -39 & -26 & 43 & 54\\48 & 21 & 14 & -7 & -36\\3 & 1 & -6 & 1 & -6\end{matrix}\right],
\ b = \left[\begin{matrix}-24\\68\\268\\-412\\-110\end{matrix}\right]. 
 \end{align*}

Вариант N 167
Дана СЛАУ $AX = b$,
Проверить совместность по теореме Кронекера-Капелли. Если СЛАУ совместна, проверить единственность решения.
Для соответствующей однородной СЛАУ проверить существование нетривиального решения. В случае, если оно существует,
найти размерность пространства решений и составить ФСР и общее решение однородной  и неоднородной СЛАУ.


\begin{align*}
 A = \left[\begin{matrix}6 & 0 & 0 & 9 & 1\\19 & -40 & 35 & -4 & -11\\119 & -200 & 175 & 16 & -51\\-71 & 200 & -175 & 56 & 59\\-237 & 600 & -525 & 132 & 173\end{matrix}\right],
\ b = \left[\begin{matrix}-22\\327\\1547\\-1723\\-5081\end{matrix}\right]. 
 \end{align*}

Вариант N 168
Дана СЛАУ $AX = b$,
Проверить совместность по теореме Кронекера-Капелли. Если СЛАУ совместна, проверить единственность решения.
Для соответствующей однородной СЛАУ проверить существование нетривиального решения. В случае, если оно существует,
найти размерность пространства решений и составить ФСР и общее решение однородной  и неоднородной СЛАУ.


\begin{align*}
 A = \left[\begin{matrix}-3 & 6 & 5 & 6 & 7\\-1 & 9 & -9 & 9 & 7\\-16 & 60 & -16 & 60 & 56\\-8 & -12 & 56 & -12 & 0\\-8 & -6 & -4 & 9 & 4\end{matrix}\right],
\ b = \left[\begin{matrix}-23\\-32\\-220\\36\\81\end{matrix}\right]. 
 \end{align*}

Вариант N 169
Дана СЛАУ $AX = b$,
Проверить совместность по теореме Кронекера-Капелли. Если СЛАУ совместна, проверить единственность решения.
Для соответствующей однородной СЛАУ проверить существование нетривиального решения. В случае, если оно существует,
найти размерность пространства решений и составить ФСР и общее решение однородной  и неоднородной СЛАУ.


\begin{align*}
 A = \left[\begin{matrix}1 & -6 & -7 & -5 & -2\\-2 & -3 & 5 & -9 & 3\\11 & -6 & -41 & 21 & -18\\2 & -5 & 6 & -3 & -5\\6 & -5 & -6 & -4 & 5\end{matrix}\right],
\ b = \left[\begin{matrix}-16\\-60\\192\\-16\\-85\end{matrix}\right]. 
 \end{align*}

Вариант N 170
Дана СЛАУ $AX = b$,
Проверить совместность по теореме Кронекера-Капелли. Если СЛАУ совместна, проверить единственность решения.
Для соответствующей однородной СЛАУ проверить существование нетривиального решения. В случае, если оно существует,
найти размерность пространства решений и составить ФСР и общее решение однородной  и неоднородной СЛАУ.


\begin{align*}
 A = \left[\begin{matrix}-7 & 3 & 8 & 3 & -2\\-16 & -20 & 4 & -20 & -12\\-92 & -68 & 48 & -68 & -56\\36 & 92 & 16 & 92 & 40\\136 & 264 & 16 & 264 & 128\end{matrix}\right],
\ b = \left[\begin{matrix}77\\-80\\-12\\628\\1576\end{matrix}\right]. 
 \end{align*}

Вариант N 171
Дана СЛАУ $AX = b$,
Проверить совместность по теореме Кронекера-Капелли. Если СЛАУ совместна, проверить единственность решения.
Для соответствующей однородной СЛАУ проверить существование нетривиального решения. В случае, если оно существует,
найти размерность пространства решений и составить ФСР и общее решение однородной  и неоднородной СЛАУ.


\begin{align*}
 A = \left[\begin{matrix}-6 & -2 & 3 & 8 & 1\\-56 & -44 & 20 & 4 & 28\\-248 & -184 & 92 & 48 & 116\\200 & 168 & -68 & 16 & -108\\624 & 512 & -216 & 16 & -328\end{matrix}\right],
\ b = \left[\begin{matrix}-34\\-40\\-296\\24\\208\end{matrix}\right]. 
 \end{align*}

Вариант N 172
Дана СЛАУ $AX = b$,
Проверить совместность по теореме Кронекера-Капелли. Если СЛАУ совместна, проверить единственность решения.
Для соответствующей однородной СЛАУ проверить существование нетривиального решения. В случае, если оно существует,
найти размерность пространства решений и составить ФСР и общее решение однородной  и неоднородной СЛАУ.


\begin{align*}
 A = \left[\begin{matrix}-5 & -3 & -8 & -7 & -5\\7 & 3 & 8 & -6 & -4\\20 & 6 & 16 & -51 & -35\\-50 & -24 & -64 & 9 & 5\\-7 & 1 & -2 & -5 & 2\end{matrix}\right],
\ b = \left[\begin{matrix}14\\50\\292\\-208\\-32\end{matrix}\right]. 
 \end{align*}

Вариант N 173
Дана СЛАУ $AX = b$,
Проверить совместность по теореме Кронекера-Капелли. Если СЛАУ совместна, проверить единственность решения.
Для соответствующей однородной СЛАУ проверить существование нетривиального решения. В случае, если оно существует,
найти размерность пространства решений и составить ФСР и общее решение однородной  и неоднородной СЛАУ.


\begin{align*}
 A = \left[\begin{matrix}-5 & 3 & 5 & -7 & 9\\-8 & 4 & -3 & 3 & -9\\-55 & 29 & 0 & -6 & -18\\25 & -11 & 30 & -36 & 72\\6 & 2 & 4 & -6 & -6\end{matrix}\right],
\ b = \left[\begin{matrix}-12\\128\\604\\-676\\-22\end{matrix}\right]. 
 \end{align*}

Вариант N 174
Дана СЛАУ $AX = b$,
Проверить совместность по теореме Кронекера-Капелли. Если СЛАУ совместна, проверить единственность решения.
Для соответствующей однородной СЛАУ проверить существование нетривиального решения. В случае, если оно существует,
найти размерность пространства решений и составить ФСР и общее решение однородной  и неоднородной СЛАУ.


\begin{align*}
 A = \left[\begin{matrix}-9 & -8 & 6 & 9 & 2\\3 & -3 & 3 & -1 & -4\\-24 & -44 & 36 & 32 & -8\\-48 & -20 & 12 & 40 & 24\\-6 & 0 & -1 & 9 & -6\end{matrix}\right],
\ b = \left[\begin{matrix}195\\36\\924\\636\\124\end{matrix}\right]. 
 \end{align*}

Вариант N 175
Дана СЛАУ $AX = b$,
Проверить совместность по теореме Кронекера-Капелли. Если СЛАУ совместна, проверить единственность решения.
Для соответствующей однородной СЛАУ проверить существование нетривиального решения. В случае, если оно существует,
найти размерность пространства решений и составить ФСР и общее решение однородной  и неоднородной СЛАУ.


\begin{align*}
 A = \left[\begin{matrix}4 & -4 & 3 & 4 & 5\\-8 & 5 & -3 & 7 & -4\\-16 & 4 & 0 & 44 & 4\\48 & -36 & 24 & -12 & 36\\-6 & -3 & 7 & 5 & 9\end{matrix}\right],
\ b = \left[\begin{matrix}-21\\75\\216\\-384\\-50\end{matrix}\right]. 
 \end{align*}

Вариант N 176
Дана СЛАУ $AX = b$,
Проверить совместность по теореме Кронекера-Капелли. Если СЛАУ совместна, проверить единственность решения.
Для соответствующей однородной СЛАУ проверить существование нетривиального решения. В случае, если оно существует,
найти размерность пространства решений и составить ФСР и общее решение однородной  и неоднородной СЛАУ.


\begin{align*}
 A = \left[\begin{matrix}-4 & -7 & -3 & -3 & -2\\9 & -5 & -9 & 9 & 1\\20 & -48 & -48 & 24 & -4\\-52 & -8 & 24 & -48 & -12\\-1 & -1 & 4 & -2 & -6\end{matrix}\right],
\ b = \left[\begin{matrix}-18\\-13\\-124\\-20\\37\end{matrix}\right]. 
 \end{align*}

Вариант N 177
Дана СЛАУ $AX = b$,
Проверить совместность по теореме Кронекера-Капелли. Если СЛАУ совместна, проверить единственность решения.
Для соответствующей однородной СЛАУ проверить существование нетривиального решения. В случае, если оно существует,
найти размерность пространства решений и составить ФСР и общее решение однородной  и неоднородной СЛАУ.


\begin{align*}
 A = \left[\begin{matrix}-6 & 8 & 3 & 4 & -9\\-6 & 9 & -8 & 6 & 6\\12 & -21 & 49 & -18 & -57\\-3 & 7 & 2 & 2 & -5\\-8 & -9 & 6 & -7 & -1\end{matrix}\right],
\ b = \left[\begin{matrix}-144\\5\\-457\\-81\\-6\end{matrix}\right]. 
 \end{align*}

Вариант N 178
Дана СЛАУ $AX = b$,
Проверить совместность по теореме Кронекера-Капелли. Если СЛАУ совместна, проверить единственность решения.
Для соответствующей однородной СЛАУ проверить существование нетривиального решения. В случае, если оно существует,
найти размерность пространства решений и составить ФСР и общее решение однородной  и неоднородной СЛАУ.


\begin{align*}
 A = \left[\begin{matrix}7 & -8 & 0 & 2 & -9\\2 & 5 & 3 & 2 & -5\\18 & -57 & -15 & -2 & -11\\6 & -8 & 6 & -3 & 1\\-8 & -7 & -5 & -3 & -9\end{matrix}\right],
\ b = \left[\begin{matrix}-27\\-51\\147\\91\\-87\end{matrix}\right]. 
 \end{align*}

Вариант N 179
Дана СЛАУ $AX = b$,
Проверить совместность по теореме Кронекера-Капелли. Если СЛАУ совместна, проверить единственность решения.
Для соответствующей однородной СЛАУ проверить существование нетривиального решения. В случае, если оно существует,
найти размерность пространства решений и составить ФСР и общее решение однородной  и неоднородной СЛАУ.


\begin{align*}
 A = \left[\begin{matrix}-4 & -3 & 7 & -4 & 7\\-17 & -4 & 66 & 28 & 1\\-97 & -29 & 351 & 128 & 26\\73 & 11 & -309 & -152 & 16\\231 & 42 & -948 & -444 & 27\end{matrix}\right],
\ b = \left[\begin{matrix}51\\253\\1418\\-1112\\-3489\end{matrix}\right]. 
 \end{align*}

Вариант N 180
Дана СЛАУ $AX = b$,
Проверить совместность по теореме Кронекера-Капелли. Если СЛАУ совместна, проверить единственность решения.
Для соответствующей однородной СЛАУ проверить существование нетривиального решения. В случае, если оно существует,
найти размерность пространства решений и составить ФСР и общее решение однородной  и неоднородной СЛАУ.


\begin{align*}
 A = \left[\begin{matrix}-5 & -2 & -6 & 6 & -7\\-27 & -30 & -14 & -18 & -45\\-123 & -126 & -74 & -54 & -201\\93 & 114 & 38 & 90 & 159\\294 & 348 & 132 & 252 & 498\end{matrix}\right],
\ b = \left[\begin{matrix}-19\\131\\467\\-581\\-1686\end{matrix}\right]. 
 \end{align*}

Вариант N 181
Дана СЛАУ $AX = b$,
Проверить совместность по теореме Кронекера-Капелли. Если СЛАУ совместна, проверить единственность решения.
Для соответствующей однородной СЛАУ проверить существование нетривиального решения. В случае, если оно существует,
найти размерность пространства решений и составить ФСР и общее решение однородной  и неоднородной СЛАУ.


\begin{align*}
 A = \left[\begin{matrix}9 & 6 & -1 & -9 & -3\\-6 & 1 & 9 & 2 & -3\\57 & 13 & -48 & -37 & 6\\-4 & -7 & 9 & -4 & 8\\5 & 2 & -7 & -2 & 9\end{matrix}\right],
\ b = \left[\begin{matrix}-59\\-66\\153\\-28\\65\end{matrix}\right]. 
 \end{align*}

Вариант N 182
Дана СЛАУ $AX = b$,
Проверить совместность по теореме Кронекера-Капелли. Если СЛАУ совместна, проверить единственность решения.
Для соответствующей однородной СЛАУ проверить существование нетривиального решения. В случае, если оно существует,
найти размерность пространства решений и составить ФСР и общее решение однородной  и неоднородной СЛАУ.


\begin{align*}
 A = \left[\begin{matrix}-5 & -3 & -4 & 4 & -3\\-1 & 2 & -4 & -7 & 7\\-19 & -1 & -28 & -16 & 19\\-11 & -17 & 4 & 40 & -37\\0 & 5 & -5 & -6 & 2\end{matrix}\right],
\ b = \left[\begin{matrix}-21\\-31\\-187\\61\\-39\end{matrix}\right]. 
 \end{align*}

Вариант N 183
Дана СЛАУ $AX = b$,
Проверить совместность по теореме Кронекера-Капелли. Если СЛАУ совместна, проверить единственность решения.
Для соответствующей однородной СЛАУ проверить существование нетривиального решения. В случае, если оно существует,
найти размерность пространства решений и составить ФСР и общее решение однородной  и неоднородной СЛАУ.


\begin{align*}
 A = \left[\begin{matrix}2 & -6 & -8 & 6 & 2\\7 & -4 & -8 & -7 & 5\\41 & -38 & -64 & -17 & 31\\-29 & 2 & 16 & 53 & -19\\-8 & 8 & -1 & 3 & -6\end{matrix}\right],
\ b = \left[\begin{matrix}28\\-104\\-436\\604\\140\end{matrix}\right]. 
 \end{align*}

Вариант N 184
Дана СЛАУ $AX = b$,
Проверить совместность по теореме Кронекера-Капелли. Если СЛАУ совместна, проверить единственность решения.
Для соответствующей однородной СЛАУ проверить существование нетривиального решения. В случае, если оно существует,
найти размерность пространства решений и составить ФСР и общее решение однородной  и неоднородной СЛАУ.


\begin{align*}
 A = \left[\begin{matrix}-2 & -6 & 7 & 2 & 3\\24 & -12 & 0 & 8 & 4\\88 & -72 & 28 & 40 & 28\\-104 & 24 & 28 & -24 & -4\\-304 & 96 & 56 & -80 & -24\end{matrix}\right],
\ b = \left[\begin{matrix}23\\-216\\-772\\956\\2776\end{matrix}\right]. 
 \end{align*}

Вариант N 185
Дана СЛАУ $AX = b$,
Проверить совместность по теореме Кронекера-Капелли. Если СЛАУ совместна, проверить единственность решения.
Для соответствующей однородной СЛАУ проверить существование нетривиального решения. В случае, если оно существует,
найти размерность пространства решений и составить ФСР и общее решение однородной  и неоднородной СЛАУ.


\begin{align*}
 A = \left[\begin{matrix}-7 & 4 & 2 & -4 & -5\\-13 & -29 & 53 & 24 & -10\\-93 & -129 & 273 & 104 & -70\\37 & 161 & -257 & -136 & 30\\139 & 467 & -779 & -392 & 110\end{matrix}\right],
\ b = \left[\begin{matrix}-56\\-339\\-1919\\1471\\4637\end{matrix}\right]. 
 \end{align*}

Вариант N 186
Дана СЛАУ $AX = b$,
Проверить совместность по теореме Кронекера-Капелли. Если СЛАУ совместна, проверить единственность решения.
Для соответствующей однородной СЛАУ проверить существование нетривиального решения. В случае, если оно существует,
найти размерность пространства решений и составить ФСР и общее решение однородной  и неоднородной СЛАУ.


\begin{align*}
 A = \left[\begin{matrix}-2 & -3 & 6 & 4 & 7\\-51 & -14 & 28 & -13 & -4\\-261 & -79 & 158 & -53 & 1\\249 & 61 & -122 & 77 & 41\\753 & 192 & -384 & 219 & 102\end{matrix}\right],
\ b = \left[\begin{matrix}18\\529\\2699\\-2591\\-7827\end{matrix}\right]. 
 \end{align*}

Вариант N 187
Дана СЛАУ $AX = b$,
Проверить совместность по теореме Кронекера-Капелли. Если СЛАУ совместна, проверить единственность решения.
Для соответствующей однородной СЛАУ проверить существование нетривиального решения. В случае, если оно существует,
найти размерность пространства решений и составить ФСР и общее решение однородной  и неоднородной СЛАУ.


\begin{align*}
 A = \left[\begin{matrix}3 & 6 & -1 & -6 & 8\\2 & -2 & -9 & -5 & -3\\4 & 32 & 32 & -4 & 44\\8 & 8 & 7 & 3 & 4\\-4 & -8 & -2 & 7 & 6\end{matrix}\right],
\ b = \left[\begin{matrix}-21\\104\\-500\\-108\\-46\end{matrix}\right]. 
 \end{align*}

Вариант N 188
Дана СЛАУ $AX = b$,
Проверить совместность по теореме Кронекера-Капелли. Если СЛАУ совместна, проверить единственность решения.
Для соответствующей однородной СЛАУ проверить существование нетривиального решения. В случае, если оно существует,
найти размерность пространства решений и составить ФСР и общее решение однородной  и неоднородной СЛАУ.


\begin{align*}
 A = \left[\begin{matrix}1 & 2 & 8 & -9 & 3\\38 & 31 & -1 & -17 & 39\\193 & 161 & 19 & -112 & 204\\-187 & -149 & 29 & 58 & -186\\-564 & -453 & 63 & 201 & -567\end{matrix}\right],
\ b = \left[\begin{matrix}-86\\-508\\-2798\\2282\\7104\end{matrix}\right]. 
 \end{align*}

Вариант N 189
Дана СЛАУ $AX = b$,
Проверить совместность по теореме Кронекера-Капелли. Если СЛАУ совместна, проверить единственность решения.
Для соответствующей однородной СЛАУ проверить существование нетривиального решения. В случае, если оно существует,
найти размерность пространства решений и составить ФСР и общее решение однородной  и неоднородной СЛАУ.


\begin{align*}
 A = \left[\begin{matrix}6 & 1 & 0 & 6 & 5\\-9 & -2 & 6 & 2 & -4\\69 & 14 & -30 & 14 & 40\\3 & -2 & -4 & -4 & 6\\-9 & 8 & -3 & -5 & 6\end{matrix}\right],
\ b = \left[\begin{matrix}47\\-110\\738\\66\\15\end{matrix}\right]. 
 \end{align*}

Вариант N 190
Дана СЛАУ $AX = b$,
Проверить совместность по теореме Кронекера-Капелли. Если СЛАУ совместна, проверить единственность решения.
Для соответствующей однородной СЛАУ проверить существование нетривиального решения. В случае, если оно существует,
найти размерность пространства решений и составить ФСР и общее решение однородной  и неоднородной СЛАУ.


\begin{align*}
 A = \left[\begin{matrix}-9 & -9 & -2 & 6 & -9\\-6 & 9 & 3 & -1 & 5\\-66 & 9 & 7 & 19 & -11\\-6 & -81 & -23 & 29 & -61\\-9 & 1 & 2 & -8 & -5\end{matrix}\right],
\ b = \left[\begin{matrix}-134\\41\\-331\\-741\\-4\end{matrix}\right]. 
 \end{align*}

Вариант N 191
Дана СЛАУ $AX = b$,
Проверить совместность по теореме Кронекера-Капелли. Если СЛАУ совместна, проверить единственность решения.
Для соответствующей однородной СЛАУ проверить существование нетривиального решения. В случае, если оно существует,
найти размерность пространства решений и составить ФСР и общее решение однородной  и неоднородной СЛАУ.


\begin{align*}
 A = \left[\begin{matrix}4 & 7 & -1 & -5 & 7\\-2 & 5 & 0 & -9 & -6\\8 & 48 & -4 & -56 & 4\\24 & 8 & -4 & 16 & 52\\-4 & 0 & -3 & 5 & -3\end{matrix}\right],
\ b = \left[\begin{matrix}-9\\-77\\-344\\272\\-13\end{matrix}\right]. 
 \end{align*}

Вариант N 192
Дана СЛАУ $AX = b$,
Проверить совместность по теореме Кронекера-Капелли. Если СЛАУ совместна, проверить единственность решения.
Для соответствующей однородной СЛАУ проверить существование нетривиального решения. В случае, если оно существует,
найти размерность пространства решений и составить ФСР и общее решение однородной  и неоднородной СЛАУ.


\begin{align*}
 A = \left[\begin{matrix}1 & -9 & 2 & 3 & 7\\-7 & 0 & 7 & -3 & 0\\-25 & -27 & 34 & -3 & 21\\31 & -27 & -22 & 21 & 21\\6 & -1 & 9 & 1 & 9\end{matrix}\right],
\ b = \left[\begin{matrix}8\\-68\\-248\\296\\-19\end{matrix}\right]. 
 \end{align*}

Вариант N 193
Дана СЛАУ $AX = b$,
Проверить совместность по теореме Кронекера-Капелли. Если СЛАУ совместна, проверить единственность решения.
Для соответствующей однородной СЛАУ проверить существование нетривиального решения. В случае, если оно существует,
найти размерность пространства решений и составить ФСР и общее решение однородной  и неоднородной СЛАУ.


\begin{align*}
 A = \left[\begin{matrix}7 & 0 & 2 & 9 & -5\\63 & 10 & 28 & 71 & -45\\343 & 50 & 148 & 391 & -245\\-287 & -50 & -132 & -319 & 205\\-889 & -150 & -404 & -993 & 635\end{matrix}\right],
\ b = \left[\begin{matrix}21\\269\\1429\\-1261\\-3867\end{matrix}\right]. 
 \end{align*}

Вариант N 194
Дана СЛАУ $AX = b$,
Проверить совместность по теореме Кронекера-Капелли. Если СЛАУ совместна, проверить единственность решения.
Для соответствующей однородной СЛАУ проверить существование нетривиального решения. В случае, если оно существует,
найти размерность пространства решений и составить ФСР и общее решение однородной  и неоднородной СЛАУ.


\begin{align*}
 A = \left[\begin{matrix}-3 & -9 & 8 & -2 & -1\\5 & 4 & -1 & -5 & -8\\8 & -20 & 28 & -28 & -36\\-32 & -52 & 36 & 12 & 28\\8 & 4 & 6 & -8 & -2\end{matrix}\right],
\ b = \left[\begin{matrix}30\\21\\204\\36\\68\end{matrix}\right]. 
 \end{align*}

Вариант N 195
Дана СЛАУ $AX = b$,
Проверить совместность по теореме Кронекера-Капелли. Если СЛАУ совместна, проверить единственность решения.
Для соответствующей однородной СЛАУ проверить существование нетривиального решения. В случае, если оно существует,
найти размерность пространства решений и составить ФСР и общее решение однородной  и неоднородной СЛАУ.


\begin{align*}
 A = \left[\begin{matrix}8 & 8 & -6 & -4 & 5\\24 & 68 & -32 & -36 & 40\\128 & 304 & -152 & -160 & 180\\-64 & -240 & 104 & 128 & -140\\-224 & -752 & 336 & 400 & -440\end{matrix}\right],
\ b = \left[\begin{matrix}-44\\-156\\-800\\448\\1520\end{matrix}\right]. 
 \end{align*}

Вариант N 196
Дана СЛАУ $AX = b$,
Проверить совместность по теореме Кронекера-Капелли. Если СЛАУ совместна, проверить единственность решения.
Для соответствующей однородной СЛАУ проверить существование нетривиального решения. В случае, если оно существует,
найти размерность пространства решений и составить ФСР и общее решение однородной  и неоднородной СЛАУ.


\begin{align*}
 A = \left[\begin{matrix}-1 & 5 & -3 & 5 & -3\\-3 & -9 & -4 & -3 & -4\\12 & 60 & 11 & 30 & 11\\7 & -4 & -7 & -4 & 9\\-2 & -8 & 0 & 3 & 9\end{matrix}\right],
\ b = \left[\begin{matrix}-2\\-61\\299\\96\\-29\end{matrix}\right]. 
 \end{align*}

Вариант N 197
Дана СЛАУ $AX = b$,
Проверить совместность по теореме Кронекера-Капелли. Если СЛАУ совместна, проверить единственность решения.
Для соответствующей однородной СЛАУ проверить существование нетривиального решения. В случае, если оно существует,
найти размерность пространства решений и составить ФСР и общее решение однородной  и неоднородной СЛАУ.


\begin{align*}
 A = \left[\begin{matrix}1 & 4 & -5 & 6 & 3\\-3 & 8 & -6 & 0 & 4\\18 & -28 & 15 & 18 & -11\\-4 & 8 & 5 & 6 & -6\\-8 & 0 & 8 & 2 & 8\end{matrix}\right],
\ b = \left[\begin{matrix}30\\-5\\115\\54\\-122\end{matrix}\right]. 
 \end{align*}

Вариант N 198
Дана СЛАУ $AX = b$,
Проверить совместность по теореме Кронекера-Капелли. Если СЛАУ совместна, проверить единственность решения.
Для соответствующей однородной СЛАУ проверить существование нетривиального решения. В случае, если оно существует,
найти размерность пространства решений и составить ФСР и общее решение однородной  и неоднородной СЛАУ.


\begin{align*}
 A = \left[\begin{matrix}3 & -5 & -4 & -3 & -2\\5 & -5 & -1 & -2 & 5\\-8 & 0 & -12 & -4 & -28\\-7 & -7 & -3 & 1 & -5\\8 & -4 & -8 & 7 & 4\end{matrix}\right],
\ b = \left[\begin{matrix}49\\-18\\268\\72\\4\end{matrix}\right]. 
 \end{align*}

Вариант N 199
Дана СЛАУ $AX = b$,
Проверить совместность по теореме Кронекера-Капелли. Если СЛАУ совместна, проверить единственность решения.
Для соответствующей однородной СЛАУ проверить существование нетривиального решения. В случае, если оно существует,
найти размерность пространства решений и составить ФСР и общее решение однородной  и неоднородной СЛАУ.


\begin{align*}
 A = \left[\begin{matrix}2 & -1 & 5 & 3 & 9\\-3 & 8 & -7 & 4 & 2\\18 & -35 & 43 & -7 & 19\\6 & 9 & -1 & 2 & -9\\5 & 4 & 9 & -6 & 3\end{matrix}\right],
\ b = \left[\begin{matrix}71\\31\\89\\-32\\-18\end{matrix}\right]. 
 \end{align*}

Вариант N 200
Дана СЛАУ $AX = b$,
Проверить совместность по теореме Кронекера-Капелли. Если СЛАУ совместна, проверить единственность решения.
Для соответствующей однородной СЛАУ проверить существование нетривиального решения. В случае, если оно существует,
найти размерность пространства решений и составить ФСР и общее решение однородной  и неоднородной СЛАУ.


\begin{align*}
 A = \left[\begin{matrix}-9 & -1 & -1 & 3 & 4\\8 & -3 & -3 & 5 & 8\\-76 & 11 & 11 & -13 & -24\\-2 & -1 & 6 & -1 & 5\\-9 & 3 & 0 & 8 & 9\end{matrix}\right],
\ b = \left[\begin{matrix}-16\\15\\-139\\-54\\-88\end{matrix}\right]. 
 \end{align*}

\end{document}